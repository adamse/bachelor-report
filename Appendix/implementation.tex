\section{Constructing Agents}
\label{ap:constructing_experiments}
Due to the complexity of testing the implementations it is hard to directly find the source of the faulty behaviour. In this project a lot of time has been spent tracking down errors in the code and two possible reasons for this problem has been identified.

The first bottleneck for the project was that during the implementation phase some of the details regarding the algorithm was unclear. When the errors in the code started appear it was hard to track down were they had occurred and many hours were spent on debugging. If it would just have been a logical error it would have been easier to track down by stepping through the code step by step. Although, when not understanding the algorithm it becomes harder, a lot of time were spent trying to find out if it was a logical error or if the problem was that we implemented the algorithm wrong. Even though the implementation phase started with creating a high level description some details were left unclear, it still proved helpful during the high level reasoning going back and forward, trying to figure out the next step and what could be wrong with the current implementation.

There were problem finding a proper structure and design for the code, much because the lack of structure in the RL-Glue(\ref{ap:eval_glue}) library. The lacking structure of the code made it harder to test independent pieces of the algorithms and validate the correctness of the pieces. It was not until the end of the phase the code started to become more structured and therefor it was easier to verify individual pieces of the code. The main reason to the \note{unstable code base} is the rapid prototyping used when creating the agents, resulting in a working solution but maybe not the best way to do it. One might think that due to the small code base used in the project, the structure and design of the code is not as relevant as in a larger project. However, the structure and design still plays an important part and makes it easier to identify errors and simple mistakes.