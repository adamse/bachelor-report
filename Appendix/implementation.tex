\section{Constructing Agents}
\label{ap:constructing_experiments}
Due to the complexity of testing the implementations of the agents, it was hard to directly find the source of the faulty behaviour. In this project a lot of time has been spent tracking down errors in the code and two possible reasons for this problem has been identified.

The first problem was frequent during the implementation phase due to some details regarding the algorithm was unclear. Therby when errors in the code started to appear, it was hard to find the root of the problem. If it would have been a logical error, the error could have been tracked down by stepping through the code step by step. Although, when not fully comprehending the algorithm it became harder and thereby a lot of time were spent trying to find out source of the problem. Even though the implementation phase started with creating a high level description some details were left unclear. However, when summarizing it still proved helpful during the high level reasoning going back and forward, trying to figure out the next step and what could be wrong with the current implementation.

The second reason was due to the fact that there were a problem finding a proper structure and design for the code. The lacking structure of the code made it harder to test independent pieces of the algorithms and validate the correctness of the pieces. It was not until the end of the phase the code started to become more structured and therefore it became easier to verify individual pieces of the code. The main reason to the lacking quality of the code was the rapid prototyping used when the agents were created, resulting in a working solution but it was far from the best way to implement the feature. One might think that due to the small code base used in the project, the structure and design of the code is not as relevant as in a larger project. Nevertheless, the structure and design still plays an important part and makes it easier to identify errors and simple mistakes.