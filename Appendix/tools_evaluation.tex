\section{Programming Language}
\label{ap:prog_lang_eval}
To implement the algorithms the Java programming language was used as stated in section \ref{sec:prog_env}. A common rule when developing any kind of software is to use the right tool for the job, which is also a frequent comment when discussing which is the best programming language. Java is great for larger projects due to the fact that it is a statically typed and a object oriented programming language. Nonetheless, it also required a lot of boilerplate code for achieving the smallest of results. 

For this reason it would be more suitable to use a dynamic programing language, for example Python. Using a more dynamic language is more suitable for rapid prototyping and experimenting in a way that Java can never be. Throughout the project the confidence in the choice of programming language decreased. The lack of support to easily implement the data structures needed for this project, lead to increased complexity in the code base and more error prone code. Instead of focusing on the core algorithm, focus was shifted towards trying to work around the problem with data structures and trying to create easier abstractions. 

A problem that presented itself in the DBN-\etre algorithm when implementing the conditional probability table. The problem is when indexing with multiple keys and that for every level another table needed to be implemented, creating a chained and complex structure. To clarify HashMaps\footnote{HashMap http://docs.oracle.com/javase/7/docs/api/java/util/HashMap.html} was used for the implementation, resulting in a nested structure with a new level of HashMaps for every level and quickly getting out of hand. This is just one of many examples from the code created during the implementation phase. Nonetheless, while the need for abstractions is obvious it may very well be the fact it will prove to equally cumbersome to implement in another language.

To conclude some of the data structures proves to be rather complex to implement easily and use and may not be at the fault of Java, but when solely implementing algorithms the language should be a tool and barely noticeable and not subject to discussion and extra work. A possible solution to the problem with data structures for the conditional probability table is to instead use a data structure resembling a relationship database might be a better match, nevertheless the same kind of behaviour is easy to recreate in a programming language.