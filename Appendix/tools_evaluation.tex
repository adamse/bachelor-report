\section{Programming Language}
\label{ap:prog_lang_eval}
To implement the algorithms the Java programming language was used as stated in section \ref{sec:prog_env}. A common rule when developing any kind of software is to use the right tool for the job, which is also a frequent comment when discussing which is the best programming language. Java is great for larger projects due to the fact that it is a statically typed and a object oriented programming language. Nonetheless, it also required a lot of boilerplate code for achieving the smallest of results. Throughout the project the confidence in the choice of programming language decreased. Instead of focusing on the core algorithm, focus was shifted towards trying to work around the problem with data structures and trying to create easier abstractions in Java. For this reason it would be more suitable to use a dynamic programing language, for example Python. A dynamic programming language is more suitable for rapid prototyping and experimenting in a way that Java can never be. 