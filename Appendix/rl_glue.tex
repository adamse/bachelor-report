\section{RL-Glue Framework}
\label{ap:eval_glue}
Although RL-Glue provides the end user with many features and offers an easy interface for creating reinforcement learning experiments, there are drawbacks with the framework. One major problem with the Java version of the RL-Glue framework is the lack of examples and documentation of the framework. Documentation of how the classes and their methods are supposed to work and interact with each-other is more or less non-existing, drastically increasing the level of effort needed to start working with RL-Glue. We found that the best way to start using RL-Glue was to either extend existing examples or inspecting the source of RL-Glue trying to figure out how the authors thought when they constructed the framework.

Another problem with the structure of the framework is that it does not use conventions of the Java programming language\footnote{http://www.oracle.com/technetwork/java/javase/documentation/codeconvtoc-136057.html}. By creating classes that are responsible for holding data or information about an specific state and not implementing comparisons between these classes. It may seem as a technicality to not use the conventions for the language, however by not fulfilling the contracts for when designing a class, the error that occur due to contract is time spent wasted tracking down. Therefore we used an variation of the Facade and Adapter programming patterns \parencite{gamme1994designpatterns} to work around the shortcomings of the framework in an manageable and structured way.