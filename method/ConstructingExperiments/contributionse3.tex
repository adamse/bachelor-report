\subsection{Implementation choices and extensions for \etre}
\label{sec:e3_our_contribution}

\paragraph{Planning in dynamic bayesian networks}

The DBN-\etre\ algorithm does not in itself define what algorithm should be
used for planning when the MDP is structured as a DBN
\parencite{kearns1999efficient}. It considers planning a black box, leaving the
choice of planning algorithm to the implementers. 

Value iteration can be done with a factored representation of an MDP in a
fairly straightforward manner. The same equations that normal value iteration
(section~\ref{sec:valueiteration}) is based on can be used when the MDP is
factored too. The only difference is that in order to calculate the probability
of a state transition, $P(s'| a, s)$ one has to find the product of all the
partial transitions,
\begin{equation}
  \prod\limits _{i} P\left(s_i' | a, pa(s_i)\right)
\end{equation}
where $i$ ranges over all partial states and $pa(s_i)$ is the setting of the
parents of the partial state $s_i$.

When an MDP has this structure, observations of partial transitions can be
pooled together when the state variables are part of similar structures in the
MDP. In the version of DBN-\etre\ described here, all state variables that have
the same number of parent variables have their observations pooled together. 

\paragraph{One policy per state variable}
\label{sec:one_policy_per_state_variable}

For some MDPs it is possible to compute a separate policy for each state
variable individually. This is the case when there is a separate action taken
for each state variable, which is true for the Invasive Species environment
(section \ref{sec:experiment_env}). In the implementation of \etre\ used in
this thesis, this policy computation is performed in two steps. 

In the first step, a policy is calculated for state variables that have no
other state variables than themselves as parents in the DBN, and these states
are marked as done. This calculation is done by value iteration where the reward function
is described in equation \eqref{equation:fusk} and described below. Since there is now a decided action for each value for
these state variables, the transition probabilities for these variables can be
considered as pure Markov chains in the next step. In this step, a policy is
found for state variables whose parents are marked as done, until all state
variables are done.  In this second step, the transition probabilities of the
parents are thus treated as independent of the action taken.

The reward function for the partial action $a(i)$ and the partial state $s(i)$ in the Invasive Species environment can be described as follows:

\begin{align}
\label{equation:fusk}
R(s(i),a(i)) = \, 
 & c(s(i)) r_c + t(s(i)) r_t  \nonumber \\
 & + n(s(i)) r_n + e(s(i)) r_e   \nonumber \\
 & +  x(a(i)) t(s(i)) r_x + p(a(i)) e(s(i)) r_p
\end{align}

Here $c(s(i))$ is equal to 1 if $s(i)$ is infected and 0 otherwise, $r_c$ is the reward given for each infected reach, $t(s(i))$ is the number of tamarix-infested habitats in $s(i)$, $r_t$ is the reward given for each tamarix-invaded habitat, $n(s(i))$ is the number of habitats with native trees in $s(i)$, $r_n$ is the reward given for each native habitat, $e(s(i))$ is the number of empty habitats in $s(i)$, $e_r$ is the reward given for empty habitats, $x(a(i))$ is equal to 1 if the action taken is to exterminate tamarix trees and otherwise 0, $r_x$ is the reward for each tamarisk tree that the agent tries to remove, $p(a(i))$ is equal to one if the action taken is to plant native trees and 0 otherwise, and $r_p$ is the reward for each empty slot that the agent tries to plant a native tree in. 
Since equation \eqref{equation:fusk} is a simple linear equation, the unknown variables ($r_i, r_t, r_x$ and $r_p$) can be calculated exactly once a few data points have been collected. Once this is done, the agent can use equation \eqref{equation:fusk} to calculate the reward for any partial state-action pair. 

Planning for each state variable individually has the benefit of making the
planning algorithm linear in the number of state variables, greatly reducing
the time needed to calculate a policy. However, there are several downsides to
using this kind of approximation, some of which are discussed in section
\ref{sec:e3_factored_discussion}. 

