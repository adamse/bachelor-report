\section{Algorithm implementation}
\label{sec:implementation}

The experiment utilizes RL-Glue (section \ref{sec:rl_glue}) for connecting the
agents and environment for the experiment. Thereby using the pre-defined
specifications of RL-Glue when developing the agents, improving their
re-usability. 

The most complex phase of constructing the agents was the verification of their
behaviour. The difficulty of this is correlated with the size of the problem
domain and for a smaller problem the behaviour is easier to verify. For this 
environments were constructed specifically for testing the agents, an
important design consideration was to make sure that the correct results were
easy to either derive or were obvious from inspection. By starting with smaller
problems and using an iterative approach it was possible to identify
bottlenecks in our implementations and correct possible errors early. The two
following sections describe the GridWorld and network simulator enviornments
that were used for this purpose. 

\subsection{GridWorld}

The GridWorld environment was implemented to easily be able to verify the
correctness of the MBIE algorithm. It consists of a grid of twelve squares with
one blocked square, one starting square, one winning square, one losing square and
eight empty squares. The agent can take five actions, north, south, west,
east or exit. The exit-action is only possible from the winning or losing
state. When taking an action being in one state and the action is directed to
another empty state there is an 80\% probability to succeed and 10\%
probability to fail and 10\% to go sideways.

\subsection{Network simulator}

A simple computer network simulation was implemented as a simple test for
the \etre\ algorithm. In this environment, the agent tries to keep a network of
computers up and running. All computers start in the running state, but there
is a chance that they randomly stop working. If a computer is down, it has a
chance to cause other computers connected to it to also fail. In each time
step, the agent chooses one computer to restart, which with 100 percent
probability will be in working condition in the next time step. The agent is
rewarded for each computer in the running state after each time step. 
