\section{Environment specification}
\label{sec:experiment_env}

For the experiment, the invasive species environment from the 2014 edition of
the Reinforcement Learning Competition was used. The environment is a
simulation of an invasive species problem, in this case a river network with
invading species where the goal of the agent is to eradicate unwanted species
while replanting native species. 

The environment's model of the river network has parameters, such as the size
of the river network and the rate at which plants spread, which can be
configured in order to create different variations of the environment.  The
size of the river network is defined by two parameters: the number of reaches
and the number of habitats per reach. A habitat is the smallest unit of land
that is considered in the problem. A habitat can either be invaded by the
tamarix, which is an unwanted species, empty or occupied by native species. A
reach is a collection of neighboring habitats. The structure of the river
network is defined in terms of which reach is connected to which
\parencite{invasiveSpecis2014:Online}. In figure \ref{fig:river} a model of a
river network is shown.

\begin{figure}[ht]
\centering
\includegraphics[width=0.9\textwidth]{images/river_network.pdf}
\caption{A river network, as modelled by the invasive species reinforcement learning environment.}
\label{fig:river}
\end{figure}

There are four possible actions (eradicate tamarisks, plant native trees,
eradicate tamarisks and plant native trees, and finally a wait-and-see action),
and the agent chooses one of these actions per reach per time step. What
actions are available to the agent depends on the state of each reach. It is
always possible to choose the wait-and-see action, but there has to be one
or more tamarisk-invaded habitats in a reach for the eradicate or eradicate and
plant actions to be available and there has to be at least one empty habitat in
a reach for the plant-native-trees action to be avalable
\parencite{invasiveSpecis2014:Online}. 
