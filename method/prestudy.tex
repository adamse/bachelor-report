\section{Pre-study \note{Ny text}}
\label{sec:pre_study}
\label{sec:select_agents}
The project started with a substantial pre-study of the previous research work in the reinforcement learning field. This was done with the purpose of getting a clear view of the state of current reinforcement learning research. Since there exists a rather large scope of techniques, there was a need to limit the field of study quickly in order to get a manageable body of text to work with. 

\todo[inline]{Add examples of different kind of techniques, making the section more real}
With the Invasive Species environment in mind the focus shifted towards selecting agents to test in the environment. In order to achieve this, studies in research areas regarding algorithms with focus on large state spaces were collected and reviewed. To be able to collect significant results and observations the algorithms need to make use of different techniques for handling the complexity of problems with large state spaces. The next requirement when selecting agents is an approximation of the complexity of the technology used in the agent and if it is even feasible to implement the agent in reasonable time. The most critical factors is that the technique used in the algorithm is part of an interesting research topic, meaning that research is being conducted but the problem is not completely solved.