\subsection{Constructing Agents \note{Ny text}}
\label{sec:trans_math_intruc}
\todo[inline]{Be more specific? Refering to the structure of direct reports used?}
Depending on the type of algorithm to implement, there were different kinds of obstacles to overcome. Not all research papers describe a complete implementation of the algorithm covered in the paper, instead they only explain specific parts of a Reinforcement Learning algorithm \parencite{kearns2002near}. In order to use such papers, additional work is required to combine concepts of separate algorithms and adapt them to work properly together.

Another issue arises when a  research paper introduces an algorithm but does not clearly describe the algorithm with pseudo code or high-level flow charts. Rather, the article may contain mathematical reasoning about general strategies used and verifications of these by formal proofs. When this is the case, a step-by-step procedure translating the mathematics and concepts from the research papers into executable code was followed.  

This step-by-step procedure could be descrbied as follows. The first step was to form an idea about how the agent is supposed to work. From this idea, a list of steps was extracted and used as a starting point and as material for high-level discussions about the agent. From the high-level discussion more and more concrete instructions was derived. The last step was to implement the mathematical formulas and analyze the best way to implement them using our chosen programming language. 