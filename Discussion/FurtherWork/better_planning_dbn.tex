\subsection{Better Planning Algorithm for Factored \etre\ in Markov Decision Processes \note{N,K}}
\label{sec:better_planing_algos}
% To be able to completely take advantage of the factored structure, the complete process needs to be using factored versions. The second algorithm studied in this thesis, DBN-\etre\ is utilizing a dynamic Bayesian Network in order to factor the the representation of the environment. In section \ref{sec:factored_e3}, there were a discussion regarding which planning algorithm to use. In the end a slightly modified version of Value Iteration was used, due to its simplicity and given the scope of the project. However as mentioned in the beginning of this section, the entire process from representation to planning needs utilize factored versions of the algorithm in order to maximize the output. Before the decision about using the modified Value Iteration, there were a discussion among alternative strategies to use in order to create an factored(approximate) value function and the two main options going forward are briefly presented below.
In order to being able to completely take advantage of the factored structure, the complete process needs to be using factored versions. DBN-\etre\ is utilizing a dynamic Bayesian Network in order to factor the representation of the environment. In section \ref{sec:factored_e3}, there were a \note{discussion} regarding which planning algorithm to use. In the end a slightly modified version of Value Iteration was used, due to its simplicity and given the scope of the project. 
% Hör ihop med stycket ovan. Har svårt att förstå första meningen.
However, as mentioned in the beginning of this section, the entire process from representation to planning needs utilize factored versions of the algorithm in order to maximize the output. Before the decision to use the modified Value Iteration, there were a discussion among alternative strategies to use in order to create an factored(approximate) value function and the two main options going forward are briefly presented below.



% Inte så goa titlar på paragraferna nej

\paragraph{Approximate Value Determination}
Utilises a value determination algorithm to optimally approximate a value function, for factored representations using dynamic Bayesian Networks. The algorithm is using linear programming in order to achieve as good approximation as possible, over the factors associated with small subsets of problem features. However, as the authors of the algorithm mentions, their algorithm does not take advantage of the structured \note{CPTs}. Which leaves room for further improvements using dynamic programming steps \parencite{koller1999computing}. Therefore it could be a good start when starting to improve the DBN-\etre\ algorithm used in this thesis and step by step improve its efficiency. 

\paragraph{Approximate Value Function}
Takes a different approach than the others options discussed, but the it attempts to solve the same problem, planning in large state spaces. This method represents the approximation of the value functions as a linear combination of basis functions. Each basis involves a small subset of the environment variables. A strength is that the algorithm comes in both an linear and dynamic programming versions. It could be more complex than the second option to implement, however \textcite{guestrin2003efficient} presents results for problems with $10^{40}$ states. Which may very well result in a bigger improvement than the previous options discussed.


