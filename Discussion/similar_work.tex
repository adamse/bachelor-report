\section{Similar studies \note{N, K}}
The work with reducing the complexity of the reinforcement learning problems with environments containing large state spaces is not a new research topic. However, this thesis is taking a rather uncommon approach to the problem. Instead of trying to derive new algorithms for solving the problem, the focus is instead focused on usefulness of research conducted within the area of reinforcement learning problems. To be even more specific this thesis has zoomed in on two possible attempts on solving the problem with environments containing large state spaces. An approach that is rather uncommon and thus similar studies is as well uncommon. 
%Behöver skrivas om, tydligare
Thereby this section will combining focus on both studies similar to this thesis comparing the usefulness of research already conducted along with a summarized study of studies of work being done to reduce the complexity of environments with large state spaces.

\todo[inline]{Samma miljö kanske, dess pac?}

\paragraph{Similar Studies Regarding Comparing Existing Research}
\todo[inline]{Borde finnas återkoppling i diskussion kring detta}
It has been shown in \parencite{strehl2004empirical} that MBIE outperforms \etre\ in the MDPs RiverSwim and SixArms. \parencite{dietterich2013pac}


\paragraph{Large State Spaces using Dynamic Bayesian Networks}
This paragraph is devoted to similar methods using dynamic Bayesian networks as an underlying representation in order to model the structure of the environment. Due to the relevance of using dynamic Bayesian network as a method for tackling similar problems it is thesis on its own in order to summarize them all. It is also possible to study section \ref{sec:better_planing_algos} for further references, however their main focus on the planning algorithm.

An algorithm to DBN-\etre\ is the algorithm by \textcite{ross2012model} which is as well utilizing a factored representation for the underlying structure. However, in the DBN-\etre\ algorithm no planning algorithm was specified and therefore leaving an empty square in the implementation affecting the overall complexity of the algorithm. In the algorithm by \textcite{ross2012model} 

\todo[inline]{Resultatjämförelse}

\paragraph{Similar Studies Regarding Large State Spaces Using Model Based Something grejen}
Grattis MBIE, er punkt.