\subsection{Planning algorithm for DBN-\etre\ in factored MDPs}
\label{sec:better_planing_algos}

To be able to completely take advantage of the factored structure of an MDP, the complete
process needs to use factored versions. DBN-\etre\ utilizes a dynamic
bayesian network in order to factor the representation of the environment. In
section \ref{sec:factored_e3} a slightly modified version of value iteration is
used due to its simplicity and the scope of the project. Below are other
possible planning algorithms for factored MDPs.

\paragraph{Approximate value determination}

Utilizes a value determination algorithm to optimally approximate a value
function, for factored representations using dynamic bayesian networks. The
algorithm uses linear programming in order to achieve as good an approximation
as possible, over the factors associated with small subsets of problem
features. However, as the authors of the algorithm mentions, their algorithm
does not take advantage of the factored conditional probability tables. This
leaves room for further improvements using dynamic programming steps
\parencite{koller1999computing}.

\paragraph{Approximate value functions}

This method represents the approximation of the value functions as a linear
combination of basis functions. Each basis involves a small subset of the
environment variables. A strength is that the algorithm comes in both a linear
and dynamic programming versions. It could be more complex than approximate
value determination to implement. \textcite{guestrin2003efficient} presents
results for problems with $10^{40}$ states, which may very well result in a
bigger improvement than the previous options discussed.
