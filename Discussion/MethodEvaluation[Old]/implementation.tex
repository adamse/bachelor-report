\subsection{Constructing Experiments \note{B,K}}
\label{sec:eval_implementation}
In section \ref{sec:implementation} it is stated that smaller environments were used to easily debug the code and behaviour of the agents. This technique has been very effective and almost critical to move the project forward. It is hard to write code, especially without being able to verify it along with the problem with the design mentioned earlier in this section. By using smaller environments it became easier to verify the behaviour of the agent and reason why it was malfunctioning sometimes. Along with the possibility to start measuring the impact performance and memory wise when testing new optimization ideas. Compared with charging headless straight into the problems with large state spaces and trying to figure out why our agents did not perform as expected. \note{One example of how we did this is that we constructed our own MDP GridWorld, consisting of 4X3-1 states, with one winning state and one losing state.}
\todo[inline]{Här kan ni skriva om gridworld om ni känner för det}
