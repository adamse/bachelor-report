\subsection{MBIE \note{N}}
\paragraph{General shape od} In comparison to the \etre\ performance graphs, the MBIE performance exhibits a much smoother transition from poor to good performance. This is due to the fact that MBIE does not have a clear distinction between exploration and exploitation in phases. Instead, MBIE in effect always gives state-action pairs that are relatively unexplored a bonus to their expected value in order to promote exploration.  

In the graphs for MBIE there are several ``dips'' in performance. These could be explained as cases when the algorithm by chance enters previously unexplored states and spends several steps exploring this and similar/adjacent states. 

\paragraph{Realistic MBIE and original MBIE}
The realistic version outperforms the original MBIE in every test. The reason for this is probably 

%utforska runt "nya" outforskade states

% Formen annorlunda än 
%jämföra realistisk/orealistisk
% - realistisk: slutar utforska tidigare, samma orsak som beskrivs tidigare (lätt att hitta mest värdefulla statet). 
% - 