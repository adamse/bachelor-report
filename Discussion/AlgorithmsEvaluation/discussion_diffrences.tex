\subsection{DBN-\etre\ vs MBIE \note{N}}
\todo[inline]{Perhaps belongs in conclusion?}
An expectation on both agents is that they should converge on optimal behavior as $t$ goes to infinity. However, it is clear from the results presented in chapter \ref{ch:results} that neither version of the MBIE agent reaches the same level of performance as the DBN-\etre\ agent. In the smaller problem as seen for example in figure~\ref{fig:tests1}b the diffrence is not all that diffrent between DBN-\etre\ agent and the realistic MBIE. Nevertheless, it is still clear that DBN-\etre\ is far superior in finding the converging policy. Previous studies have been concluded as a comparisson regarding the original \etre\ algorithm and MBIE, in \textcite{strehl2004empirical} that MBIE outperforms \etre\ in their testing environments. The results in this thesis utilized a heavily optimized version of the original \etre\ agent using factored representation and the results speak for themselfs mentioned earlier in this report. 

\paragraph{Unfair comparisons} In one sense, the comparison between our implementations of MBIE and DBN-\etre\ are not very fair. The DBN-\etre\ implementation has been heavily optimized to work with factored MDPs and the invasive species environment in particular, whereas the MBIE implementation is much more generalized. The discussions regarding the generality of the agents in furher continued in section \ref{sec:impact_of_one_env}. However, as mentioned in the last paragraph of section \ref{sec:e3_factored_discussion} when the structure of the underlying MDP the benefits of using a factored representations provides a big advantage as seen in the figures in chapter \ref{ch:results}. This also increases the responsibility on the designer of the factored representation and if done in the wrong way it will prove to become a bottleneck instead of a performance increase.

\paragraph{Large state spaces}
When aprouching large state spaces it is still visible that the algorithms work. Comparing the 4 reaches and 3 habitats per reach case seen in figure~\ref{fig:tests3}a with the 5 reaches and 3 habitats per reach case seen in figure~\ref{fig:tests2}b this can be seen. The algorithms do still improve over time although taking longer time to converge which is expected as the state space increases in size. Futhermore, the run time for the algorithms increased notably with larger state spaces but is nothing we have looked deeper into.  

A noteable difference between the different algorithms is that the realistic version of MBIE performs about the same as the original MBIE in larger state spaces compared to smaller state spaces where the realistic MBIE were clearly better than the original, compare figure~\ref{fig:tests1}b with figure~\ref{fig:tests2}b to see this difference. 

