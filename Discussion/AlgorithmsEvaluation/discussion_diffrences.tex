\subsection{DBN-\etre\ vs MBIE \note{N}}
\todo[inline]{Perhaps belongs in conclusion?}

An expectation on both agents is that they should converge on optimal behavior
as $t$ goes to infinity. However, it is clear from the results presented in
chapter \ref{ch:results} that neither version of the MBIE agent reaches the
same level of performance as the DBN-\etre\ agent. In the smaller problem as
seen for example in figure~\ref{fig:tests1}b the diffrence is not all that
diffrent between DBN-\etre\ agent and the realistic MBIE. Nevertheless, it is
still clear that DBN-\etre\ is far superior in finding the converging policy.
Previous studies have been concluded as a comparisson regarding the original
\etre\ algorithm and MBIE, in \textcite{strehl2004empirical} that MBIE
outperforms \etre\ in their testing environments. The results in this thesis
utilized a heavily optimized version of the original \etre\ agent using
factored representation and the results speak for themselfs mentioned earlier
in this report. 

\paragraph{Unfair comparisons}

In one sense, the comparison between our implementations of MBIE and DBN-\etre\
are not very fair. The DBN-\etre\ implementation has been heavily optimized to
work with factored MDPs and the invasive species environment in particular,
whereas the MBIE implementation is much more generalized. The discussions
regarding the generality of the agents in furher continued in section
\ref{sec:impact_of_one_env}. However, as mentioned in the last paragraph of
section \ref{sec:e3_factored_discussion} when the structure of the underlying
MDP the benefits of using a factored representations provides a big advantage
as seen in the figures in chapter \ref{ch:results}. This also increases the
responsibility on the designer of the factored representation and if done in
the wrong way it will prove to become a bottleneck instead of a performance
increase.

\paragraph{Large state spaces}

When the number of states are increased it is clear that the algorithms still
work. Comparing the 4 reaches and 3 habitats per reach case seen in
figure~\ref{fig:tests3}a with the 5 reaches and 3 habitats per reach case seen
in figure~\ref{fig:tests2}b this can be seen. The algorithms do still improve
over time although taking longer time to converge, which is expected as the
state space increases in size. Futhermore, the policy calulation run time for
the algorithms increased notably with larger state spaces but not to such an
extent that it posed a practical issue.

A noteable difference between the original and realistic versions of MBIE is
that the realistic version has the same level of performance as the original
MBIE in larger state spaces whereas in smaller state spaces the realistic MBIE
was clearly better than the original. Compare figure~\ref{fig:tests1}b with
figure~\ref{fig:tests2}b to see this difference. 

A practical limitation today is the amount of space needed to keep all the
states in the memory of the computer. With the constant increase of memory in
today's computers its becomming less and less of a serious limitation. However,
the problem arises when the information is needed as a result of updates
resulting in slower lookup time when the state space increases
\parencite{szepesvari2010algorithms}. A problem that we have experienced is
that the running time becomes a problem before running out of memory with large
state spaces. 

First of, MBIE approaches this problem by introducing confidence intervals on
the values of state and then when sampling moving this shrinking this interval
closing in on the actual value. By doing so the algorithm have an estimation of
reality early on even with few samples in a large state space and continues to
improve on this estimaton in comparising to other algorithms
\todo{example / source} that require a much larger sample size in order
to give any result at all. 

Secondly, \todo[inline]{Daniel kan du berätta lite om minnesproblemen för oss är du snäll?}
