\section{Using models for simulating real world problems}
\label{sec:ethics_inv_spec}

This thesis evaluates and compares two existing algorithms and therefore it has
a negligible impact on matters such as ethics, social or economics in today's
society. Therefore, this section focuses on the further impact of reinforcement learning using models of the real world and simulations. The environment Invasive Species is a simulation of the problem with invasive
species. This domain focuses on the problem where a spreading process needs to be controlled in a
river network with native and invading plant species
\parencite{invasiveSpecis2014:Online}. 

It is commonly known how fragile ecosystems are to changes. It is a complex
system where one change could help the system in the short term while damaging it in the long
term. Therefore using simulations with self-learning algorithms allows to test
more methods than time and money would allow in the real world. The simulation is a rough model
of the real world and it is hard to capture all elements of the real-world
problem. Therefore, the answers from the simulation will also at best be roughly
correct.

This presents a practical use of reinforcement learning which simplified can be
viewed as a smart trial and error algorithm for finding good strategies for
working with real world problems like invasive species.
