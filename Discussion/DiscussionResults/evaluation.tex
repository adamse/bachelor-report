\subsection{Testing methodology}

Evaluating agents is hard due to the process of chosing appropriate parameters
for both the algorithm and the environment. For example it is uncertain how
much the paramaters for the algorithms affect the outcome of the experiment and
trying out all possible combination is not feasible within the scope of this
thesis. However, a possible solution for this problem is to derive the optimal
paramaters using mathmatical proofs.

The same problem appears when chosing the parameters for the invasive species
environment. The obvious problem is if the paramaters were chosen diffrently
would the DBN-\etre\ algorithm perform in the same way or if the optimizations
discussed in section \ref{sec:e3_factored_discussion} would not perform in
another setting. Nevertheless, with the MBIE algorithm there is a greater
probability that the results will be comparable to the results collected.

The last potential issue of the evaluation is a rather complex issue. Is the
policy the agents finds an optimal policy for the environment?  When the
number of habitats and reaches in the invasive species environment increases it
becomes impossible to check manually of the policy computed by the algorithm is
correct. \textcite{dietterich2013pac} mentions that they are uncertain if
their algorithm achieves an optimal solution and this should be taken into
consideration when evaluating the agents in this thesis.
