\subsection{Evaluation of experiments \note{N}}
The evaluation of the algorithms is critical for achieving good results and as well deriving conslusions regarding the results collected. Evaluating agents is a tricky process due to the process of chosing approximatly correct parameters for both the algorithm and the environment. To be more specific for example it is uncertain how much the paramaters for the MBIE affect the outcome of the experiment and trying out every diffrent combination is not possible within the scope of this theses. However, a possible solution for this problem is to derive the optimal paramaters using mathmatical proofs.

The same problem with the existed when chosing the problems for the environment, Invasive Species. The obvious problem is if the paramaters were chosen diffrently would the DBN-\etre\ algorithm perform in the same way or if the optimizations discussed in section \ref{sec:e3_factored_discussion} would not perform in another possible setup. However, the MBIE algorithm there is a greater probability the results will be comparable to the results collected and discussed earlier due to the disussion in \ref{sec:impact_of_one_env}.

The last potential issue of the evaluation is a rather complex issue. Is the solution solution the agents finding, an optimal solution for the problem environment. 
har vi en optimal lösning? pac-dietrish