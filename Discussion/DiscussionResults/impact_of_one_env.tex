\subsection{Impact of using one environment \note{N}}
\label{sec:impact_of_one_env}
A pressing issue is that, due to the limitations of the project, the results presented in this report are only collected from one environment. This presents a practical difficulty, considering that it is challenging to evaluate and verify the generality of the algorithms. From the the results of the DBN-\etre\ algorithm there are indications that the optimizations applied to the algorithm may be the specific for the Invasive Species environment. Which reduces the creditability regarding the generality conclusions presented in this thesis. Due to the fact that it is not possible within the scope of this thesis to evaluate the performance of the DBN-\etre\, without the optimizations applied. 

Even though developed in parallel the generality of MBIE is more credible due to the fact it was simultaneously tested alongside invasive species in a simpler environment, GridWorld(section \ref{sec:intro_grid_world}). Thereby more or less, forcing the generality during construction of the agent. However even though the generality was partly forced during development, the lack of verification is hard to overlook.