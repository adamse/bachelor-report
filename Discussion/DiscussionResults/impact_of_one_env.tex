\subsection{Impact of using one environment \note{N}}
\label{sec:impact_of_one_env}
A pressing issue is that, due to the limitations of the project, the results presented in this report are only collected from one environment. This presents a practical difficulty regarding the problem statement, considering that it is impossible evaluate the generality of the algorithms implemented and tested in this study. From the the results of the DBN-\etre\ algorithm there are indications that the optimizations applied to the algorithm may be the specific for the Invasive Species environment. Which reduces the creditability regarding the generality conclusions presented in this thesis. The issue presented itself in the fact there is no way within the scope of this thesis to evaluate the performance of the agent without the optimizations applied. 

Nevertheless, as discussed in section \ref{sec:e3_factored_discussion} the DBN-\etre\ algorithm converge quick to an optimal policy, which takes a considerate longer time for the MBIE algorithm. Even though developed in parallel the generality of MBIE is more credible due to the fact it was simultaneously tested alongside Invasive Species in a simpler environment representing a grid world. Forcing the generality during construction of the agent. However even though the generality was partly forced during development, the lack of verification is hard to overlook.