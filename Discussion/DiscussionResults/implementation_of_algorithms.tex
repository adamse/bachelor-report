\subsection{Implementation of algorithms \note{N}}
One of the biggest challenges when constructing and evaluating algorithms is to validate the actual implementation of the algorithm, especially when there is no key to compare the result against. When building upon the work of another creator there is always a possibilities that specifications were interpreted poorly or mistakes during the implementation lead to less efficient or even wrong solution. The lack of similar work containing results also makes it hard to get an estimate of how well our algorithms do in comparison to what the methods they use are expected to achieve.

In order to increase the credibility of the implementations used one method is to perform unit testing of the implementation. By creating implementations which enables unit testing it becomes easier to test the individual pieces of the algorithms and thereby verify correct behaviour. For more regarding information the implementation process see appendix \ref{ap:implementation_process}. Comparing the process of using automated tests with manually validating the results of the complete algorithms behaviour as described in section \ref{sec:implementation}, the advantages are obvious.
\todo[inline]{Kanske inrelevant, kanske inte - kanske behöver referense}