\subsection{Implementation of algorithms \note{N}}
One of the biggest challenges when constructing and evaluating algorithms is to validate the actual implementation of the algorithm and in addition there is no key to compare the result against. When building upon the work of another creator there is always a possibilities that specifications were interpreted poorly or mistakes during the implementation lead to less efficient or even wrong solution. As mentioned earlier in this paragraph, due to the lack of similar work containing results it is also hard to get an estimate of how well our algorithms actually perform in comparison with what they actually could have achieved. 

In order to increase the credibility of the implementations used one method is to perform unit testing of the implementation. By creating implementations which enables unit testing it becomes easier to test the individual pieces of the algorithms and thereby verify correct behaviour. For more regarding information the implementation process see appendix \ref{ap:implementation_process}. Compare that process of using automated tests with manually validating the results of the complete algorithms behaviour as described in section \ref{sec:implementation}, the advantages is obvious. 
\todo[inline]{Kanske inrelevant, kanske inte - kanske behöver referense}