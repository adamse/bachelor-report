% 1 Motivation
For some problems in the real world it is infeasible to optimize behavior by making a model and do calculations on it, 
in cases like these one could find good solutions by trial and error. 
Reinforcement learning is a way of finding optimal behavior by systematic trial and error.
% 2 Problem statement
This thesis aims to compare different learning techniques and evaluate those. 
Model-based interval estimation (MBIE) and dynamic Bayesian network - explicit explore or exploit (DBN-\etre)
 are two learning algorithms that is evaluated. 
% 3 Approach
To evaluate the techniques we built agents using the algorithms and 
then simulated these in the environment invasive species.
% 4 Results
DBN-\etre\ is better than MBIE at finding an optimal or near optimal policy in the 
environment invasive species with the environment parameters that we used.

% 5 Conclusions
Using a factored model like DBN-\etre\ show certain advantages operating in 
the factored environment invasive species. For example it achieves a near optimal
policy within fewer episodes than MBIE.


% Följande saker ska vara med i en abstract Motivation, Problem Statement, Approach, REsults

% Motivation
% Why do we care about the problem and the results? 
% If the problem isn't obviously "interesting" it might be better to
% put motivation first; but if your work is incremental progress on
% a problem that is widely recognized as important, then it is 
% probably better to put the problem statement first to indicate which
%  piece of the larger problem you are breaking off to work on.
% This section should include the importance of your work, the difficulty
% of the area, and the impact it might have if successful.

% Problem statement
% What problem are you trying to solve? What is the scope of your work 
%(a generalized approach, or for a specific situation)? Be careful not to 
% use too much jargon. In some cases it is appropriate to put the problem 
% statement before the motivation, but usually this only works if most 
% readers already understand why the problem is important.

% Approach
% How did you go about solving or making progress on the problem? Did you 
% use simulation, analytic models, prototype construction, or analysis of
% field data for an actual product? What was the extent of your work
% (did you look at one application program or a hundred programs in
% twenty different programming languages?) What important variables did
% you control, ignore, or measure?

% Results
% What's the answer? Specifically, most good computer architecture papers 
% conclude that something is so many percent faster, cheaper, smaller, 
% or otherwise better than something else. Put the result there, in
% numbers. Avoid vague, hand-waving results such as "very", "small",
% or "significant." If you must be vague, you are only given license
% to do so when you can talk about orders-of-magnitude improvement. 
% There is a tension here in that you should not provide numbers that
% can be easily misinterpreted, but on the other hand you don't have
% room for all the caveats.

% Conclusions
% What are the implications of your answer? Is it going to change the 
% world (unlikely), be a significant "win", be a nice hack, or simply 
% serve as a road sign indicating that this path is a waste of time 
% (all of the previous results are useful). Are your results general, 
% potentially generalizable, or specific to a particular case?
