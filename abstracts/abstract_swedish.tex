% 1 Motivation
För vissa verklighetsbaserade optimeringsproblem där det gäller att finna ett bästa beteende är det orimligt söka efter det optimala beteendet genom att skapa en modell av problemet och sedan utföra beräkningar på den. När så är fallet, kan man hitta bra lösningar genom trial and error. Reinforcement learning är ett sätt att hitta optimalt beteende genom att systematisk pröva sig fram.
% 2 Problem statement
Detta kandidatarbete har som mål att jämföra olika inlärningstekniker och utvärdera dessa.
Model-based interval estimation (MBIE) och Explicit Explore or Exploit med dynamiska bayesnät (DBN-\etre) är 
två självlärande algoritmer som här utvärderas.
% 3 Approach
För att utvärdera de olika teknikerna implementerades agenter som använde algoritmerna, och dessa testkördes sedan 
i miljön Invasive Species från tävlingen Reinforcement Learning Competition.
% 4 Results
DBN-\etre\ är bättre än MBIE på att hitta en optimal eller nära optimal policy i Invasive Species med valda parametrar.
% 5 Conclusions
Att använda en faktoriserad modell, så som DBN-\etre\, visar på tydliga fördelar i faktoriserade miljöer, som i till exempel Invasive Species. 
Ett exempel på detta är att den når en nästan optimal policy på färre episoder än MBIE.
