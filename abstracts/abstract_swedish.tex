% 1 Motivation
För vissa verklighetsbaserade optimeringsproblem där det gäller att finna ett bästa beteende är det orimligt söka efter det optimala beteendet genom att skapa en modell av problemet och sedan utföra beräkningar på den. I vissa sådana fall kan man hitta bra lösningar genom trial and error. Reinforcement learning är ett sätt att hitta optimalt beteende genom att systematisk pröva sig fram.
% 2 Problem statement
Detta kandidatarbete har som mål att jämföra olika inlärningstekniker och utvärdera dessa.
Model-based interval estimation (MBIE) och Explicit Explore or Exploit med dynamiska bayesnät (DBN-\etre) är 
två självlärande algoritmer som här utvärderas.

% 3 Approach
För att utvärdera de olika teknikerna implementerades agenter som använde algoritmerna, och dessa testkördes sedan 
i miljön invasive species från tävlingen Reinforcement learning competition.

% 4 Results
DBN-\etre\ är bättre än MBIE på att hitta en optimal eller nära optimal policy i miljön invasive species med valda parametrar.
% 5 Conclusions
Att använda en faktoriserad model, så som DBN-\etre\, visar på tydliga fördelar i faktoriserade miljöer, som i till exempel invasive species. 
Ett exempel på detta är att den når en nästan optimal policy på färre episoder än MBIE.

% Följande saker ska vara med i en abstract Motivation, Problem Statement, Approach, REsults

% Motivation
% Why do we care about the problem and the results? 
% If the problem isn't obviously "interesting" it might be better to
% put motivation first; but if your work is incremental progress on
% a problem that is widely recognized as important, then it is 
% probably better to put the problem statement first to indicate which
%  piece of the larger problem you are breaking off to work on.
% This section should include the importance of your work, the difficulty
% of the area, and the impact it might have if successful.

% Problem statement
% What problem are you trying to solve? What is the scope of your work 
%(a generalized approach, or for a specific situation)? Be careful not to 
% use too much jargon. In some cases it is appropriate to put the problem 
% statement before the motivation, but usually this only works if most 
% readers already understand why the problem is important.

% Approach
% How did you go about solving or making progress on the problem? Did you 
% use simulation, analytic models, prototype construction, or analysis of
% field data for an actual product? What was the extent of your work
% (did you look at one application program or a hundred programs in
% twenty different programming languages?) What important variables did
% you control, ignore, or measure?

% Results
% What's the answer? Specifically, most good computer architecture papers 
% conclude that something is so many percent faster, cheaper, smaller, 
% or otherwise better than something else. Put the result there, in
% numbers. Avoid vague, hand-waving results such as "very", "small",
% or "significant." If you must be vague, you are only given license
% to do so when you can talk about orders-of-magnitude improvement. 
% There is a tension here in that you should not provide numbers that
% can be easily misinterpreted, but on the other hand you don't have
% room for all the caveats.

% Conclusions
% What are the implications of your answer? Is it going to change the 
% world (unlikely), be a significant "win", be a nice hack, or simply 
% serve as a road sign indicating that this path is a waste of time 
% (all of the previous results are useful). Are your results general, 
% potentially generalizable, or specific to a particular case?
