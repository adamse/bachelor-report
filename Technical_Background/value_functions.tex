\section{Basic algorithms for solving MDPs}

A policy $\pi$ is a function from a state $s$ to an action $a$ that operates in
the context of a Markov decision process, i.e. $\pi \colon S \to A$. A policy
is thus a description of how to act in each state of an MDP. An arbitrary
policy is denoted by $\pi$ and the optimal policy (the policy with the maximal
utility in the MDP) is denoted by $\pi^*$. The rest of this section
describes some basic algorithms for solving an MDP, that is to say, finding an optimal
policy for it \parencite{barto1998reinforcement}.

\subsection{Value functions}

To solve an MDP, most algorithms use an estimation of values of states or
states and actions. Two value functions are usually defined, the state-value function $V :
S \to \mathbb R$ and the state-action-value function $Q : S \times A \to
\mathbb R$. As the names imply, $V$ signifies how good a state is, while $Q$
signifies how good an action in a state is. The state-value function,
$V^\pi(s)$, returns the expected value when starting in state $s$ and following
policy $\pi$ thereafter. The state-action-value function, $Q^\pi(s, a)$, returns
the expected value when starting in state $s$ and taking action $a$ and
thereafter following policy $\pi$. The value functions for the optimal policy, the policy
that maximizes the utility, 
are denoted by $V^*(s)$ and $Q^*(s, a)$ \parencite{barto1998reinforcement}. 

Equations \eqref{equation:v} and \eqref{equation:q} show the state-value
function $V$ and the state-action-value function $Q$ defined in terms of
utility (section~\ref{sec:utility}). The future states, $s_t$, are
stochastically distributed according to the transition probabilities of the MDP
and the policy $\pi$ (see section~\ref{sec:mdps}).

\begin{align}
\label{equation:v}
V_t^\pi(s) = \mathbb{E} \left\{
  \left. U_t
  \right\vert s_t = s
\right\}
\end{align}

\begin{align}
\label{equation:q}
Q_t^\pi(s, a) = \mathbb{E} \left\{
  \left. U_t
  \right\vert s_t = s, a_t = a
\right\}
\end{align}

Both $V^\pi$ and $Q^\pi$ can be estimated from experience. This can be done by
maintaining the average of the rewards that have followed each state when
following the policy $\pi$. When the number of times the state has been
encountered goes to infinity, the average over these histories of rewards
converges to the true values of the value function
\parencite{barto1998reinforcement}.

\paragraph{Using dynamic programming to find $V$ and $Q$}

Another way to find value functions is to use dynamic programming techniques. \parencite{bellman1957mdp}. Dynamic programming is a way of dividing a problem into subproblems that can be
solved independently. If the result of a particular subproblem is needed again,
it can be looked up from a table. Examples of dynamic programming algorithms are
policy iteration and value iteration, which are discussed in sections
\ref{sec:pol_itr} and \ref{sec:valueiteration}. These algorithms are often the
basis for more advanced algorithms, among them the ones described in chapter
\ref{ch:algo}. 

\subsection{Policy iteration}
\label{sec:pol_itr}

Policy iteration is a method for solving an MPD that will converge to an
optimal policy and the true value function in a finite number of iterations if
the process is a finite MDP. The algorithm consists of three steps:
initialization, policy evaluation and policy improvement
\parencite{barto1998reinforcement}.

\begin{description}
\item[Initialization] \hfill \\
    Start with an arbitrary policy $\pi$ and arbitrary value function $V$.

\item[Policy evaluation] \hfill \\
  Compute an updated value function, $V$, for policy $\pi$ in the MDP by using
  the update rule \eqref{eq:vfupdate} until $V$ converges. $\pi(s, a)$ is the
  probability of taking action $a$ in state $s$ using policy $\pi$.

\begin{equation} \label{eq:vfupdate}
  V_{k+1} (s) = \sum_a \pi(s, a) \sum_{s'} P(s, a, s') \left[ R(s, a) + \gamma V_k(s')  \right]
\end{equation}

\item[Policy improvement] \hfill \\
  Improve the policy by making it greedy with regard to $V$,
  equation~\eqref{eq:polimpr}\footnote{$\operatorname*{arg\,max} _a f(a)$ gives
  the $a$ that maximizes $f(a)$.}. This means that the policy will describe the
  action in each state that maximizes the expected $V$-value of the following
  state. 

\begin{equation} \label{eq:polimpr}
  \pi_{k+1} (s) = \operatorname*{arg\,max}_a \sum_{s'}P(s, a, s') \left[ R(s, a) + \gamma V(s') \right]
\end{equation}

\item Repeat evaluation and improvement until $\pi$ is stable between two iterations.
\end{description}



\subsection{Value iteration}
\label{sec:valueiteration}

Value iteration is a simplification of policy iteration where only one step of
policy evaluation is performed in each iteration. Value iteration does not compute an actual
policy until the value function has converged \parencite{barto1998reinforcement}. Value iteration works as follows:

\begin{description}
\item[Initialization] \hfill \\
    Start with an arbitrary value function $V_0$.
\item[Value iteration] \hfill \\
    Update the value function for each state using the update rule \eqref{eq:valiter}.

\begin{equation} \label{eq:valiter}
V_{k+1}(s) = \max_a \sum_{s'}{P(s, a, s') \left[R(s, a) + \gamma V_k(s')\right]}
\end{equation}

\item Repeat value iteration until $V$ converges and set $V = V_k$. As in policy iteration, by
  convergence is meant that $|V_{k+1}(s) - V_{k}(s)| \leq \epsilon, \forall s$, where
  $\epsilon$ is a small value. 

\item Compute the policy using equation ~\eqref{eq:valiterpolicy}.

\begin{equation} \label{eq:valiterpolicy}
\pi(s) = \operatorname*{arg\,max}_a \sum_{s'}{P(s, a, s') \left[R(s, a) + \gamma V(s')\right]}
\end{equation}

\end{description}

