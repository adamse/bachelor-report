\section{Markov decision process}
\label{sec:mdps}

Within reinforcement learning, the concept of Markov decision processes (MDP) is central. An MDP is a way to model an environment where state changes in it are dependent on both random chance and the actions of an agent. An MDP is defined by the quadruple $\left( S, A, P( \cdot , \cdot, \cdot ) , R( \cdot , \cdot ) \right)$ \parencite{altman2002applications}:

\begin{description}
\item[$S$] \hfill \\ 
    A set of states representing the environment.

\item[$A$] \hfill \\ 
    A set of actions that can be taken.

\item[$P \colon S \times A \times S \to \mathbb \lbrack0, 1\rbrack$] \hfill \\ 
    A probability distribution over the transitions in the environment. This
    function describes the probability of ending up in a certain target state
    when a certain action is taken from a certain origin state. 

\item[$R \colon S \times A \to \mathbb{R}$] \hfill \\ 
    A function for the reward associated with a state transition. In some
    definitions of MDPs the reward function only depends on the state.

\end{description}

MDPs are similar to Markov chains, but there are two differences. First, there
is a concept of rewards in MDPs, which is absent in Markov chains. Second, in a
Markov chain, the only thing that affects the probabilities of transitioning to
other states is the current state, whereas in an MPD both the current state and
the action taken in that state are needed to know the probability distribution
connected with the next state \parencite{altman2002applications}.

\subsection{Return}
The concept of return is used a measurement of how well an agent performs over time. In the simplest case, the return is just the sum of all rewards received in an episode. However, in a non-episodic environment, the rewards cannot just be summed up, since the interaction between the environment and the agent can go on forever. Here the concept of a discount factor is essential, $\gamma$. This parameter is a value between 0 and 1 which describes how fast the value of expected future
rewards decays as one looks further into the future
\parencite{barto1998reinforcement}. Thus, the return  can be expressed as in equation \eqref{returnDisc} in the case that discounting is used. 


\begin{equation}
\label{equation:returnDisc}
R = \sum\limits_{k = 0}^\infty \gamma^kr_{t+k+1}
\end{equation}



\subsection{Markov property}

A defining characteristic of an MDP is the Markov property - that, given the current state of the environment, one cannot gain any more information about its future behavior by also considering the previous actions and states it has been in. This can be compared to the state of a chess game, where the positions of the pieces at any time completely summarizes everything relevant about what has happened previously in the game. That is, no more information about previous moves or states of the board is needed to decide how to play or predict the future outcome of the game (disregarding psychological factors). A chess MDP that uses a chess board as its state representation could thus be an example of an MDP with the Markov property \parencite{altman2002applications}. 
\subsection{Sparse MDPs}

If there are only a few states $s'$ for which $P(s, a, s') > 0$, the MDP is called a sparse MDP. That is to say, when an agent performs a certain action, $a$, in a certain state, $s$, the environment can only end up in a small fraction out of the total number of states. If an MDP is sparse, there are several possible optimizations that can be performed \parencite{dietterich2013pac}. The MBIE algorithm (section \ref{sec:mbie}) can be extended to utilize such optimizations, one of which is described in section \ref{sec:mbie_gt}.
\subsection{Representations}
Two ways to separate the representations of an MDP is extensional or factored representation. Depending on the problem domain it can be advantageous from the computational view to use factored representation \parencite{dean1999descision}.

\paragraph{Extensional representation}
The most straightforward way to model an MDP is called extensional representation, where the set of states and actions are enumerated directly. It is also commonly refered to as an explicit representation and closely mirrors the definition we have used so far in the report when discussing the abstract view of an MDP \parencite{dean1999descision}.

\paragraph{Factored representation} 
A factored representation of the states of an MDP often results in an more compact way of describing the set of states. Certain properties or features of the states are used to categorize the states into different sets. Then one can treat all the members of the same set in the same manner. Which properties or features are used is chosen by the algorithm designer, to fit the environment.

When the MDP is factored, it enables a factored representation of rewards, actions and other components of the MDP as well. When using a factored action representation, an action can be taken based on specific state features instead of on the whole state. If the individual actions affect relativity few features or if the effects contain regularities then using a factored representation can result in compact representations of actions \parencite{dean1999descision}. 

