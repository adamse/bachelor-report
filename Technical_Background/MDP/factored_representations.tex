\subsection{Representations}

Two ways to represent an MDP are extensional and factored
representation. Depending on the problem domain it can be advantageous from the
computational view to use factored representation
\parencite{dean1999descision}.

\paragraph{Extensional representation}

The most straightforward way to model an MDP is called extensional
representation, where the set of states and actions are enumerated directly. It
is also commonly referred to as an explicit representation and this is the definition we have used so far in the report when discussing the abstract
view of an MDP \parencite{dean1999descision}.

\paragraph{Factored representation} 
\label{sec:factored_mdp}

A factored representation of the states of an MDP often results in a more
compact way of describing the set of states.  Certain properties or features of
the states are used to categorize the states into different sets. Then one can
treat all the members of the same set in the same manner. Which properties or
features are used is chosen by the algorithm designer, to fit the environment.
Formally a state $s$ is in a state space $S = S_1 \times \cdots \times S_k$
of $k$ factors, with the state at time $t$ being $s_t = (s_t(1), \ldots,
s_t(k))$.


When the MDP is factored, it enables factored representations of states,
actions and other components of the MDP as well. When using a factored action
representation, an action can be taken based on specific state features instead
of on the whole state. If the individual actions affect relativity few features
or if the effects contain regularities then using a factored representation can
result in compact representations of actions \parencite{dean1999descision}. 
