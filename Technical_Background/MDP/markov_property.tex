\subsection{Markov property}

A defining characteristic of an MDP is the Markov property---that, given the
current state of the environment, one cannot gain any more information about
its future behavior by also considering the previous actions and states it has
been in. Equation~\eqref{eq:markov_prop} defines this formally, the probability
of state $s_{t+1}$ only depends on the previous state $s_t$ and action $a_t$.

\begin{align}
\label{eq:markov_prop}
\Pr(s_{t+1} | s_t, a_t, \ldots, s_1, a_1) 
  = \Pr(s_{t+1} | s_t, a_t) 
  = P(s_t, a_t, s_{t+1})
\end{align}

This can be compared to the state of a chess game, where the positions
of the pieces at any time completely summarizes everything relevant about what
has happened previously in the game. That is, no more information about
previous moves or states of the board is needed to decide how to play or
predict the future outcome of the game (disregarding psychological factors). A
chess MDP that uses a chess board as its state representation could thus be an
example of an MDP with the Markov property \parencite{altman2002applications}. 
