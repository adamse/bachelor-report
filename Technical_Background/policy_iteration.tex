\subsection{Policy iteration}
\label{sec:pol_itr}

Policy iteration is a method for solving an MPD that will converge to an
optimal policy and the true value function in a finite number of iterations if
the process is a finite MDP. The algorithm consists of three steps:
initialization, policy evaluation and policy improvement
\parencite{barto1998reinforcement}.

\begin{description}
\item[Initialization] \hfill \\
    Start with an arbitrary policy $\pi$ and arbitrary value function $V_0$.

\item[Policy evaluation] \hfill \\
  Compute an updated value function, $V$, for the policy $\pi$ in the MDP by using
  the update rule \eqref{eq:vfupdate} until $V$ converges. Convergence means that $|V_{k+1}(s) - V_{k}(s)| \leq \epsilon, \forall s$, where $\epsilon$ is chosen as a small value.  In the update rule, $\pi(s, a)$ is the
  probability of taking action $a$ in state $s$ using policy $\pi$.

\begin{equation} \label{eq:vfupdate}
  V_{k+1} (s) = \sum_a \pi(s, a) \sum_{s'} P(s, a, s') \left[ R(s, a) + \gamma V_k(s')  \right]
\end{equation}

\item[Policy improvement] \hfill \\
  Improve the policy by making it greedy with regard to $V$,
  equation~\eqref{eq:polimpr}\footnote{$\operatorname*{arg\,max} _a f(a)$ gives
  the $a$ that maximizes $f(a)$.}. This means that the policy will describe the
  action in each state that maximizes the expected $V$-value of the following
  state. 

\begin{equation} \label{eq:polimpr}
  \pi_{k+1} (s) = \operatorname*{arg\,max}_a \sum_{s'}P(s, a, s') \left[ R(s, a) + \gamma V(s') \right]
\end{equation}

\item Repeat evaluation and improvement until $\pi$ is stable between two iterations.
\end{description}


