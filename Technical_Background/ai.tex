\section{Artificial intelligence}


The term artificial intelligence was coined by John McCarthy who described it as ``the science and engineering of making intelligent machines'' \parencite{McCarthy2007:Online}. More specifically, it addresses creating intelligent computer machines or software that can achieve specified goals computationally.  %One often categorises AI into two main branches. One branch is for problems related to thought process and logical reasoning where success can be measured against human performance. In contrast, the second branch is more focused on the behavior of the machine, which can be measured against the concept of doing the “right thing”, given the information it currently has access to \parencite{russel2002ai}. 
These goals can comprise anything, e.g. writing poetry, playing complex games such as chess or diagnosing diseases. Different branches of artificial intelligence include planning, reasoning, pattern recognition and learning from experience.

% When talking about a learning machine, we use the term agent. An agent receives information from its environment and interacts with the environment by performing different actions and evalauating the results of those actions. In the same ways as a human interacts with its environment and experiences the results of his or her actions. The environment can be thought of as a simplified version of, or a model of, reality. 




%AI has many different branches, focusing on different areas. Three examples are:
%\begin{itemize}
%\item logic - focusing on logical reasoning
%\item making observations and matching them with a pattern - e.g. facial recognition
%\item learning from experience - making use of past actions and observations in order to learn and find a way to act based on those previous observations and actions.
%\end{itemize}

%These areas comprise different subcategories and problems of their own. An approach to learning from experience is reinforcement learning, where the agent takes an action and receives a reward according to some measure of how good the action was. This could be compared to Pavlovian conditioning: teaching dogs by rewarding them for desired behavior and punishing them for undesired behavior. 
