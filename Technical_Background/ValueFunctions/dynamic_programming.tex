\subsection{Dynamic programming}

Dynamic programming is a way of dividing a problem into subproblems that can be
solved independently. If the result of a particular subproblem is needed again,
it can be looked up from a table. In reinforcement learning, dynamic
programming can be used to calculate the value functions of an MDP
\parencite{bellman1957mdp}. Examples of dynamic programming algorithms are
policy iteration and value iteration, which are discussed in sections
\ref{sec:pol_itr} and \ref{sec:valueiteration}. These algorithms are often the
basis for more advanced algorithms, among them the ones described in chapter
\ref{ch:algo}. 
