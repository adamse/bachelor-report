\subsection{Compute$\tilde{P}$}
\label{sec:ptilde}

The method for finding the sought element within the set denoted by
equation~\eqref{equation:balls_of_steels} is referred to as Compute$\tilde{P}$ in this
thesis.  The fundamental idea of the Compute$\tilde{P}$ method is that it
starts with the observed transition probabilities $\hat{P}$ and then it moves
probability mass from ``bad'' outcomes to ``good'' outcomes and finally returns
the resulting probability distribution, $\tilde{P}$. 


\paragraph{Initialization} The state transition probability distribution is
initialized according to equation~\eqref{equation:roof}, which corresponds to
the observed probabilities.

\begin{equation}
\label{equation:roof}
\tilde{P}(s'|s, a) := \hat{P}(s'|s, a) = \frac{N(s,a,s')}{N(s,a)}
\end{equation}

In equation~\eqref{equation:roof} $N(s, a, s')$ is the number of times action
$a$ has been taken in state $s$ and the agent ended up in state $s'$.

\paragraph{Moving probability mass}

The procedure of moving probability mass is done by first finding the outcome
state with the best state-value and observed probability less than 1, calling it
$\overline{s}$. Analogously the outcome with the worst state-value with an
observed probability of greater than 0 is found, and this state is called
$\underline{s}$. If a state-value has not been computed yet for a certain state,
it is assumed to have the maximum possible value. 

The probability values $\tilde{P}(\underline{s}|s,a)$ and
$\tilde{P}(\overline{s}|s,a)$ are then increased or decreased according to
equations \eqref{equation:ptilde_floor} and \eqref{equation:ptilde_roof}.

\begin{equation}
\label{equation:ptilde_floor}
\tilde{P}(\underline{s}|s,a) := \tilde{P}(\underline{s}|s,a)-\xi
\end{equation}

\begin{equation}
\label{equation:ptilde_roof}
\tilde{P}(\overline{s}|s,a) := \tilde{P}(\overline{s}|s,a)+\xi
\end{equation}

Since the sum of the probabilities needs to equal one and 
no single transition probability may fall below zero or exceed one, the probability distribution can only be modified by at most $\xi$, as given by
equation \eqref{equation:xi}, where $\Delta\omega = \omega / 2$. The variable $\Delta\omega$ denotes the total mass that can be moved, without $\tilde{P}$ having a lower chance than $1 - \delta$ of being within the confidence interval for the probability distribution. If $\xi$ is less than $\Delta \omega$, new states
$\overline{s}$ and $\underline{s}$ are found, and probabilities are moved until
mass equal to $\Delta \omega$ has been moved in total or the probability mass has all been moved to an optimal state. 
The resulting vector, $\tilde{P}$, is the one that maximizes the sum in equation \eqref{equation:q_upper}.
\begin{equation}
\label{equation:xi}
\xi = \min\{
  1 - \tilde{P}(\overline{s} | s, a)
  , \tilde{P}(\underline{s} | s, a)
  , \Delta \omega 
\}
\end{equation}
