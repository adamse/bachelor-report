\subsection{Optimistic estimations of transition probabilities}
\label{sec:computep}

The first step of the MBIE algorithm is to find optimistic estimations of
transition probabilities. Equation \eqref{equation:balls_of_steels} describes,
as a set, the confidence interval (CI) used by MBIE for the probability
distribution over destination states when taking action $a$ in state $s$. So,
the first task of the algorithm is to find the element within this set that is
maximally optimistic, meaning that it gives the best value to $(s,a)$ in the
following value-iteration step of the MBIE algorithm. A description of how this
is done in practice is found in section~\ref{sec:ptilde}. 

In equation~\eqref{equation:balls_of_steels}, $\hat{P}$ is the observed
probability distribution (treated as a vector) for destination states from
$(s,a)$, $N(s,a)$ is the number of times that the state-action pair $(s,a)$ has
been observed, $\delta$ is a confidence parameter,  $\omega$ is the value given
by equation~\eqref{equation:omega} and $\|x\|_1$ denotes the $L^1$-norm of the vector $x$.
The $L^1$-norm is the sum of the absolute value of all elements of a vector. 

\begin{align}
\label{equation:balls_of_steels}
& CI\left(\hat{P} \left| N(s, a), \delta\right.\right) \nonumber \\
& \qquad = \left\{\tilde{P} \left| \|\tilde{P} - \hat{P}\|_1 \le \omega(N(s,a), \delta), \|\hat{P}\|_1 = 1, \hat{P}_i \geq 0  \right.\right\}
\end{align}

In equation \eqref{equation:omega}, $\omega$ gives a bound for how much the
vector of transition probabilities can be changed from the observed values,
while remaining within the confidence interval. In this equation, $|S|$ is the
number of states in the MDP and the other variables have the same meaning as in
equation~\eqref{equation:balls_of_steels}. For the derivation of this equation,
see \textcite{Strehl20081309}.

\begin{equation}
\label{equation:omega}
   \omega(N(s,a),\delta) = {\sqrt{\frac{2|ln(2^{|S|}-2) - ln  \delta |}{N(s,a)}}}
\end{equation}
