\chapter{Environment and algorithms}
\label{ch:algo}

This chapter gives a description of the environment and  the algorithms that were used in the
tests described in chapter \ref{ch:method}. The main topic of the thesis is to study techniques that can be used for
reinforcement learning in large state spaces. An example of an environment that has
a large state space is Invasive Species, described in section \ref{sec:experiment_env}. 

Two algorithms are covered in this chapter, both of which deal with the problems that 
arise with large state spaces; however, they differ in the methods they apply. In 
the following chapter the general ideas behind the algorithms, as well as specific details, are
presented. 

The model-based interval estimation algorithm, described in
section~\ref{sec:mbie}, utilizes clever estimations of confidence intervals for
the Q-value functions to improve performance in sparse MDPs.
Section~\ref{sec:fac_e3} is on an algorithm that uses dynamic bayesian networks
and factored representations to improve the \etre\ algorithm to more efficiently deal with factored MDPs. 

\section{Environment specification}
\label{sec:experiment_env}

For the experiment, the invasive species environment from the 2014 edition of
the Reinforcement Learning Competition was used. The environment is a
simulation of an invasive species problem, in this case a river network with
invading species where the goal of the agent is to eradicate unwanted species
while replanting native species. 

The environment's model of the river network has parameters, such as the size
of the river network and the rate at which plants spread, which can be
configured in order to create different variations of the environment.  The
size of the river network is defined by two parameters: the number of reaches
and the number of habitats per reach. A habitat is the smallest unit of land
that is considered in the problem. A habitat can either be invaded by the
tamarix, which is an unwanted species, empty or occupied by native species. A
reach is a collection of neighboring habitats. The structure of the river
network is defined in terms of which reach is connected to which
\parencite{invasiveSpecis2014:Online}. In figure \ref{fig:river} a model of a
river network is shown.

\begin{figure}[ht]
\centering
\includegraphics[width=0.9\textwidth]{images/river_network.pdf}
\caption{A river network, as modelled by the invasive species reinforcement learning environment.}
\label{fig:river}
\end{figure}

There are four possible actions (eradicate tamarisks, plant native trees,
eradicate tamarisks and plant native trees, and finally a wait-and-see action),
and the agent chooses one of these actions per reach per time step. What
actions are available to the agent depends on the state of each reach. It is
always possible to choose the wait-and-see action, but there has to be one
or more tamarisk-invaded habitats in a reach for the eradicate or eradicate and
plant actions to be available and there has to be at least one empty habitat in
a reach for the plant-native-trees action to be avalable
\parencite{invasiveSpecis2014:Online}. 

\section{Model-based interval estimation}
\label{sec:mbie}

Model-based interval estimation is a modification of value iteration whose main feature is its  addition of confidence
intervals to the state-action values. These confidence intervals allow the agent to choose between
actions, based on how confident it is about its evaluation of them. In effect,
the less certain the agent is about its evaluation of the states and actions,
the more exploratory the actions will be. When the agent is more confident
however, it will exploit what it has learned so far about the MDP
\parencite{dietterich2013pac}.

\subsection{Value iteration with confidence intervals}
\label{sec:modification_conf_interval}

The upper bounds of the confidence intervals for the state-action values are calculated as in equation \eqref{equation:q_upper} and then, using these results, equation \eqref{equation:vMBIE} gives the state values by taking the best action for each state.

\begin{comment}
When Compute$\tilde{P}$ has been run, the resulting $\tilde{P}$ is used in a standard value
iteration, as in \eqref{equation:q_upper}. Thus, Compute$\tilde{P}$ finds the probability distribution that maximizes the sum in this equation. 

Finally, optimistic state values are computed according to \eqref{equation:vMBIE}. These values are simply the value of the best action for each state. These values are used the next time that Compute$\tilde{P}$ is run and ``good'' and ``bad'' states need to be found. 
\end{comment}

\begin{comment}
The confidence  bounds on the Q-values in the MBIE-algorithm are calculated by
making a maximally optimistic estimation of these values, given some confidence
parameter. The less times a state-action pair has been visited, the more
optimistic this estimation will be. This has the effect of promoting
exploration of actions that have been taken few times. 

When a state is first encountered by the agent, the Q-values associated with
the state are initialized with the maximum achievable reward. When the actions
are later performed, the state-action pairs have their Q-values gradually
decreased depending on the expected value. Given time, the confidence bounds will
become smaller and smaller, and the policy will converge to optimal actions
with confidence specified by a confidence parameter. The bound for the
confidence interval on a Q-value can be calculated by iterating the following
equation (cf. section~\ref{sec:valueiteration} about the basic value iteration
algorithm) for all state-action pairs until it converges:
\end{comment}

\begin{align}
\label{equation:q_upper}
Q_{upper} (s, a) = & R(s, a) + \nonumber \\
& \operatorname*{max}_{\tilde{P}(s, a)\in CI(P(s, a), \delta)} \gamma \sum_{s'} \tilde{P}(s'|s, a)\operatorname*{max}_{a'} Q_{upper}(s', a')
\end{align}

\begin{equation}
\label{equation:vMBIE}
V_{upper}(s) = \operatorname*{max}_aQ_{upper}(s,a)
\end{equation}

Section~\ref{sec:ptilde} describes a method, Compute$\tilde{P}$, for efficiently calculating the maximization step in equation \eqref{equation:q_upper}.
\subsection{Optimizations Based on Good-Turing Estimations \note{N}}
\label{sec:mbie_gt}

One problem with the method described above is that probability mass can be moved to any outcome state, without any consideration taken as to whether this outcome has ever been observed. Dietterich, Taleghan, and Crowley (2013)  make use of an optimization that deals with this by limiting the probability mass that can be moved to outcomes that have never been observed. The approximation of the probability mass in unobserved outcomes is estimated by Good and Turing as $\hat{M}_0(s,a) = |N_1(s,a)| / N(s,a)$. \parencite{gtpaper} In this equaltion, $N_1(s,a)$ is a set of the states that have been observed exactly once as an outcome when taking action $a$ in state $s$ and $N(s,a)$ is the number of times that action $a$ has been taken in state $s$ in total.

\section{\etre\ in factored Markov Decision Processes \note{N}}
\label{sec:fac_e3}

The second algorithm studied in this thesis is a version of the \etre\ algorithm that focuses on factored problem domains by modeling them as a dynamic Bayesian network. The original \etre\ algorithm is described in section \ref{sec:e3}, which gives a broad overview along with the key strategies used in the algorithm. The following section, \ref{sec:factored_e3}, considers some ways to extend the original algorithm and make use of factored representations and planning in factored domains to improve the running time of the algorithm.


\begin{longtable}{rrrrrrrr}
\resultcaption{realistic MBIE}{10}{1}
\resulthead

1 & -2\,640.85 & -2\,482.25 & -2\,218.55 & -1\,789.65 & -1\,914.25 & -2\,209.11 & 361.93  \\
2 & -1\,169.75 & -2\,266.85 & -1\,613.25 & -1\,990.25 & -920.95 & -1\,592.21 & 557.47  \\
3 & -903.45 & -842.85 & -736.75 & -1\,019.95 & -795.75 & -859.75 & 108.49  \\
4 & -913.65 & -723.25 & -746.15 & -972.75 & -807.55 & -832.67 & 107.55  \\
5 & -748.45 & -837.75 & -654.15 & -901.95 & -748.55 & -778.17 & 94.89  \\
6 & -841.25 & -826.05 & -709.15 & -736.75 & -760.35 & -774.71 & 57.03  \\
7 & -719.85 & -964.15 & -767.35 & -878.35 & -760.35 & -818.01 & 100.65  \\
8 & -844.55 & -686.15 & -711.55 & -772.15 & -760.35 & -754.95 & 61.18  \\
9 & -861.35 & -871.45 & -860.95 & -854.75 & -819.35 & -853.57 & 20.04  \\
10 & -1\,058.65 & -873.15 & -852.35 & -748.55 & -783.95 & -863.33 & 120.25  \\
11 & -726.55 & -888.45 & -801.95 & -689.55 & -807.55 & -782.81 & 77.46  \\
12 & -814.25 & -756.95 & -757.15 & -713.15 & -842.95 & -776.89 & 51.50  \\
13 & -665.85 & -920.35 & -792.55 & -807.55 & -783.95 & -794.05 & 90.38  \\
14 & -772.05 & -831.05 & -782.35 & -783.95 & -724.95 & -778.87 & 37.80  \\
15 & -736.65 & -981.05 & -779.15 & -760.35 & -972.75 & -845.99 & 120.48  \\
16 & -868.15 & -709.75 & -768.95 & -842.95 & -783.95 & -794.75 & 62.69  \\
17 & -888.25 & -863.05 & -983.75 & -866.55 & -689.55 & -858.23 & 106.30  \\
18 & -827.65 & -895.05 & -841.35 & -795.75 & -630.55 & -798.07 & 100.28  \\
19 & -745.05 & -793.95 & -721.75 & -831.15 & -654.15 & -749.21 & 68.05  \\
20 & -1\,033.25 & -1\,001.15 & -816.15 & -783.95 & -819.35 & -890.77 & 116.79  \\
21 & -797.35 & -837.85 & -747.75 & -713.15 & -736.75 & -766.57 & 50.31  \\
22 & -778.85 & -617.05 & -933.35 & -890.15 & -819.35 & -807.75 & 122.37  \\
23 & -938.85 & -915.25 & -783.15 & -866.55 & -819.35 & -864.63 & 64.74  \\
24 & -957.45 & -901.85 & -828.75 & -866.55 & -807.55 & -872.43 & 59.68  \\
25 & -758.55 & -852.95 & -800.35 & -866.55 & -760.35 & -807.75 & 50.55  \\
26 & -745.05 & -1\,053.35 & -898.75 & -960.95 & -748.55 & -881.33 & 134.57  \\
27 & -964.15 & -1\,184.85 & -743.75 & -819.35 & -913.75 & -925.17 & 168.22  \\
28 & -884.95 & -797.35 & -652.55 & -677.75 & -819.35 & -766.39 & 98.28  \\
29 & -797.35 & -1\,043.35 & -757.15 & -665.95 & -772.15 & -807.19 & 141.02  \\
30 & -826.05 & -955.75 & -837.35 & -736.75 & -878.35 & -846.85 & 79.84  \\
31 & -1\,013.05 & -834.35 & -640.75 & -724.95 & -689.55 & -780.53 & 148.21  \\
32 & -1\,009.75 & -692.85 & -653.35 & -831.15 & -736.75 & -784.77 & 142.12  \\
33 & -851.35 & -819.25 & -578.55 & -618.75 & -772.15 & -728.01 & 122.23  \\
34 & -987.85 & -868.15 & -804.35 & -772.15 & -724.95 & -831.49 & 101.74  \\
35 & -773.75 & -625.45 & -886.95 & -819.35 & -760.35 & -773.17 & 96.26  \\
36 & -932.15 & -831.05 & -734.35 & -736.75 & -701.35 & -787.13 & 94.40  \\
37 & -893.45 & -994.55 & -827.95 & -713.15 & -913.75 & -868.57 & 105.28  \\
38 & -1\,011.35 & -1\,109.05 & -688.75 & -913.75 & -831.15 & -910.81 & 162.07  \\
39 & -761.95 & -1\,019.75 & -639.95 & -972.75 & -713.15 & -821.51 & 166.15  \\
40 & -753.55 & -975.85 & -793.35 & -795.75 & -677.75 & -799.25 & 109.64  \\
41 & -920.35 & -873.15 & -766.55 & -831.15 & -783.95 & -835.03 & 63.34  \\
42 & -942.25 & -861.25 & -800.35 & -925.55 & -760.35 & -857.95 & 78.31  \\
43 & -866.45 & -746.85 & -827.95 & -842.95 & -701.35 & -797.11 & 69.92  \\
44 & -858.05 & -849.55 & -873.55 & -854.75 & -654.15 & -818.01 & 92.04  \\
45 & -815.75 & -964.15 & -780.75 & -972.75 & -772.15 & -861.11 & 99.39  \\
46 & -681.05 & -748.45 & -697.35 & -831.15 & -783.95 & -748.39 & 61.75  \\
47 & -819.25 & -874.85 & -731.95 & -665.95 & -925.55 & -803.51 & 105.21  \\
48 & -709.65 & -868.15 & -849.95 & -654.15 & -748.55 & -766.09 & 91.48  \\
49 & -879.85 & -822.55 & -805.15 & -901.95 & -937.35 & -869.37 & 55.02  \\
50 & -849.55 & -903.45 & -901.15 & -996.35 & -618.75 & -853.85 & 141.68  \\
51 & -750.15 & -881.65 & -627.35 & -913.75 & -842.95 & -803.17 & 115.86  \\
52 & -888.25 & -960.75 & -709.95 & -819.35 & -854.75 & -846.61 & 92.53  \\
53 & -955.75 & -885.05 & -733.55 & -937.35 & -842.95 & -870.93 & 88.69  \\
54 & -901.75 & -895.05 & -734.35 & -795.75 & -713.15 & -808.01 & 87.95  \\
55 & -652.45 & -989.45 & -828.75 & -819.35 & -677.75 & -793.55 & 135.65  \\
56 & -765.35 & -960.75 & -653.35 & -689.55 & -760.35 & -765.87 & 118.84  \\
57 & -964.15 & -911.95 & -781.55 & -831.15 & -949.15 & -887.59 & 78.54  \\
58 & -842.75 & -964.15 & -794.95 & -842.95 & -701.35 & -829.23 & 95.00  \\
59 & -812.55 & -900.15 & -675.35 & -748.55 & -665.95 & -760.51 & 98.16  \\
60 & -959.05 & -852.95 & -817.75 & -689.55 & -559.75 & -775.81 & 154.44  \\
61 & -746.85 & -888.35 & -864.95 & -654.15 & -842.95 & -799.45 & 97.42  \\
62 & -729.95 & -740.05 & -830.35 & -772.15 & -795.75 & -773.65 & 41.06  \\
63 & -793.95 & -903.45 & -779.15 & -972.75 & -713.15 & -832.49 & 104.02  \\
64 & -756.95 & -869.75 & -816.95 & -901.95 & -819.35 & -832.99 & 55.53  \\
65 & -859.65 & -871.45 & -919.95 & -701.35 & -724.95 & -815.47 & 96.46  \\
66 & -745.05 & -1\,007.95 & -642.35 & -748.55 & -618.75 & -752.53 & 154.39  \\
67 & -782.15 & -814.15 & -628.95 & -677.75 & -736.75 & -727.95 & 75.44  \\
68 & -768.75 & -1\,024.95 & -838.15 & -736.75 & -890.15 & -851.75 & 113.77  \\
69 & -738.35 & -810.85 & -779.15 & -724.95 & -913.75 & -793.41 & 75.35  \\
70 & -669.25 & -982.75 & -887.75 & -783.95 & -854.75 & -835.69 & 117.35  \\
71 & -743.35 & -986.05 & -917.55 & -831.15 & -890.15 & -873.65 & 91.68  \\
72 & -605.25 & -809.15 & -932.55 & -795.75 & -760.35 & -780.61 & 117.57  \\
73 & -842.85 & -864.75 & -779.95 & -689.55 & -842.95 & -804.01 & 71.40  \\
74 & -1\,033.15 & -829.35 & -781.55 & -819.35 & -677.75 & -828.23 & 129.32  \\
75 & -1\,023.15 & -891.75 & -817.75 & -807.55 & -819.35 & -871.91 & 90.97  \\
76 & -829.35 & -748.35 & -688.75 & -654.15 & -630.55 & -710.23 & 80.00  \\
77 & -740.05 & -1\,055.25 & -886.95 & -713.15 & -842.95 & -847.67 & 136.32  \\
78 & -772.05 & -814.25 & -687.15 & -866.55 & -842.95 & -796.59 & 70.60  \\
79 & -820.95 & -893.35 & -864.95 & -677.75 & -807.55 & -812.91 & 82.98  \\
80 & -718.15 & -1\,011.25 & -793.35 & -654.15 & -630.55 & -761.49 & 153.25  \\
81 & -773.75 & -802.45 & -719.35 & -772.15 & -795.75 & -772.69 & 32.65  \\
82 & -932.15 & -805.75 & -791.75 & -724.95 & -866.55 & -824.23 & 78.57  \\
83 & -987.75 & -790.65 & -711.55 & -783.95 & -795.75 & -813.93 & 103.04  \\
84 & -826.05 & -844.55 & -758.75 & -890.15 & -819.35 & -827.77 & 47.46  \\
85 & -825.95 & -675.95 & -745.35 & -890.15 & -842.95 & -796.07 & 85.06  \\
86 & -824.25 & -1\,095.65 & -860.15 & -701.35 & -701.35 & -836.55 & 161.55  \\
87 & -851.25 & -986.05 & -746.15 & -819.35 & -689.55 & -818.47 & 112.93  \\
88 & -943.95 & -954.15 & -651.75 & -748.55 & -654.15 & -790.51 & 149.94  \\
89 & -827.75 & -788.85 & -792.55 & -878.35 & -724.95 & -802.49 & 56.33  \\
90 & -987.75 & -992.85 & -593.55 & -642.35 & -866.55 & -816.61 & 189.05  \\
91 & -837.85 & -915.35 & -779.95 & -819.35 & -701.35 & -810.77 & 78.53  \\
92 & -763.65 & -858.05 & -757.15 & -807.55 & -878.35 & -812.95 & 54.51  \\
93 & -812.45 & -733.35 & -814.55 & -831.15 & -807.55 & -799.81 & 38.20  \\
94 & -814.25 & -708.05 & -1\,011.95 & -842.95 & -819.35 & -839.31 & 109.63  \\
95 & -982.75 & -937.25 & -850.75 & -866.55 & -783.95 & -884.25 & 77.47  \\
96 & -863.05 & -927.05 & -790.95 & -878.35 & -783.95 & -848.67 & 60.73  \\
97 & -969.25 & -986.05 & -757.95 & -854.75 & -689.55 & -851.51 & 129.38  \\
98 & -692.85 & -824.25 & -591.95 & -878.35 & -795.75 & -756.63 & 114.13  \\
99 & -822.65 & -911.95 & -722.55 & -748.55 & -772.15 & -795.57 & 74.78  \\
100 & -863.05 & -859.75 & -898.75 & -831.15 & -807.55 & -852.05 & 34.56  \\

\end{longtable}


\begin{longtable}{rrrrrrrr}
\resultcaption{realistic MBIE}{3}{2}
\resulthead

1 & -326.90 & -270.50 & -306.30 & -435.60 & -369.30 & -257.80 & 63.48  \\
2 & -557.90 & -162.40 & -503.80 & -651.60 & -467.60 & -468.66 & 184.70  \\
3 & -924.40 & -498.20 & -912.20 & -624.60 & -429.70 & -677.82 & 230.43  \\
4 & -1\,108.40 & -322.30 & -766.50 & -756.60 & -682.70 & -727.30 & 280.02  \\
5 & -738.30 & -578.80 & -449.80 & -714.40 & -527.80 & -601.82 & 122.90  \\
6 & -335.60 & -608.00 & -441.00 & -780.80 & -970.90 & -627.26 & 255.83  \\
7 & -255.30 & -256.10 & -300.10 & -847.10 & -837.00 & -499.12 & 313.60  \\
8 & -188.20 & -197.50 & -311.40 & -281.50 & -291.00 & -253.92 & 56.88  \\
9 & -171.40 & -241.90 & -211.80 & -223.20 & -257.10 & -221.08 & 32.75  \\
10 & -197.50 & -266.30 & -246.30 & -234.40 & -257.80 & -240.46 & 26.86  \\
11 & -174.60 & -423.20 & -183.60 & -294.30 & -403.30 & -295.80 & 117.33  \\
12 & -183.90 & -471.40 & -244.60 & -232.10 & -329.70 & -292.34 & 113.05  \\
13 & -245.60 & -256.90 & -270.80 & -170.00 & -501.30 & -288.92 & 124.98  \\
14 & -345.80 & -187.30 & -485.60 & -265.20 & -198.20 & -296.42 & 123.24  \\
15 & -275.10 & -221.10 & -455.90 & -306.40 & -169.80 & -285.66 & 108.53  \\
16 & -196.00 & -187.90 & -249.90 & -246.30 & -210.90 & -218.20 & 28.54  \\
17 & -257.60 & -255.60 & -304.60 & -242.20 & -223.60 & -256.72 & 30.01  \\
18 & -147.90 & -245.60 & -141.00 & -268.20 & -188.20 & -198.18 & 57.12  \\
19 & -254.20 & -241.60 & -247.20 & -682.60 & -270.80 & -339.28 & 192.23  \\
20 & -229.20 & -255.60 & -268.30 & -255.70 & -230.90 & -247.94 & 17.14  \\
21 & -241.50 & -284.40 & -270.80 & -285.30 & -175.50 & -251.50 & 46.03  \\
22 & -288.50 & -234.50 & -185.70 & -247.20 & -235.40 & -238.26 & 36.70  \\
23 & -298.00 & -289.10 & -282.60 & -258.10 & -259.90 & -277.54 & 17.80  \\
24 & -238.40 & -269.00 & -187.30 & -281.00 & -314.20 & -257.98 & 47.94  \\
25 & -264.00 & -336.20 & -234.50 & -198.20 & -138.90 & -234.36 & 73.60  \\
26 & -859.60 & -220.20 & -175.50 & -259.90 & -235.50 & -350.14 & 286.45  \\
27 & -221.80 & -303.30 & -225.40 & -273.50 & -259.00 & -256.60 & 34.12  \\
28 & -233.30 & -241.90 & -197.50 & -208.40 & -207.50 & -217.72 & 18.89  \\
29 & -248.10 & -199.10 & -257.20 & -235.40 & -199.10 & -227.78 & 27.30  \\
30 & -173.90 & -221.80 & -223.60 & -237.20 & -256.50 & -222.60 & 30.56  \\
31 & -210.20 & -257.40 & -212.70 & -291.00 & -282.60 & -250.78 & 37.98  \\
32 & -291.90 & -222.70 & -243.80 & -293.50 & -234.60 & -257.30 & 33.17  \\
33 & -220.90 & -209.90 & -210.90 & -256.50 & -222.90 & -224.22 & 18.96  \\
34 & -190.00 & -247.20 & -230.10 & -246.30 & -315.50 & -245.82 & 45.33  \\
35 & -270.10 & -185.10 & -234.50 & -184.80 & -290.30 & -232.96 & 48.17  \\
36 & -238.10 & -186.60 & -282.60 & -281.90 & -236.30 & -245.10 & 39.72  \\
37 & -208.40 & -188.20 & -314.60 & -327.80 & -222.70 & -252.34 & 64.21  \\
38 & -187.30 & -234.50 & -199.10 & -267.60 & -199.10 & -217.52 & 33.12  \\
39 & -185.30 & -332.50 & -235.40 & -185.70 & -212.70 & -230.32 & 60.82  \\
40 & -210.90 & -188.20 & -188.20 & -220.60 & -247.20 & -211.02 & 24.71  \\
41 & -279.20 & -161.20 & -163.70 & -330.70 & -246.30 & -236.22 & 73.76  \\
42 & -211.80 & -223.60 & -210.20 & -305.30 & -235.40 & -237.26 & 39.37  \\
43 & -207.50 & -219.50 & -174.60 & -247.20 & -255.70 & -220.90 & 32.51  \\
44 & -268.30 & -247.20 & -341.40 & -277.80 & -197.50 & -266.44 & 52.14  \\
45 & -282.60 & -222.00 & -213.60 & -306.20 & -236.30 & -252.14 & 40.30  \\
46 & -188.20 & -210.90 & -361.20 & -176.40 & -257.20 & -238.78 & 75.09  \\
47 & -223.60 & -176.40 & -223.60 & -291.70 & -292.60 & -241.58 & 50.03  \\
48 & -337.50 & -267.40 & -138.00 & -259.20 & -244.50 & -249.32 & 71.79  \\
49 & -222.00 & -259.00 & -420.20 & -187.30 & -271.00 & -271.90 & 89.17  \\
50 & -200.00 & -247.20 & -197.50 & -271.70 & -138.30 & -210.94 & 51.44  \\
51 & -163.70 & -247.20 & -176.40 & -314.60 & -196.60 & -219.70 & 61.86  \\
52 & -234.50 & -267.40 & -248.10 & -247.20 & -301.40 & -259.72 & 26.10  \\
53 & -282.60 & -244.90 & -221.10 & -249.00 & -293.70 & -258.26 & 29.55  \\
54 & -320.70 & -211.80 & -303.70 & -173.70 & -198.30 & -241.64 & 66.12  \\
55 & -561.30 & -165.50 & -199.10 & -235.40 & -270.80 & -286.42 & 158.63  \\
56 & -210.90 & -267.60 & -222.40 & -188.20 & -151.00 & -208.02 & 43.04  \\
57 & -175.50 & -281.00 & -162.80 & -163.70 & -196.60 & -195.92 & 49.48  \\
58 & -270.80 & -247.20 & -256.30 & -210.90 & -162.10 & -229.46 & 43.66  \\
59 & -211.80 & -260.80 & -188.20 & -273.50 & -237.20 & -234.30 & 34.92  \\
60 & -222.70 & -233.80 & -186.40 & -199.10 & -248.20 & -218.04 & 25.20  \\
61 & -236.30 & -258.10 & -259.00 & -161.20 & -210.20 & -224.96 & 40.83  \\
62 & -269.90 & -259.00 & -234.50 & -199.10 & -326.40 & -257.78 & 47.02  \\
63 & -317.10 & -237.20 & -188.20 & -245.40 & -173.00 & -232.18 & 56.67  \\
64 & -130.10 & -255.60 & -220.20 & -293.50 & -196.60 & -219.20 & 61.84  \\
65 & -139.20 & -211.80 & -222.70 & -256.50 & -291.00 & -224.24 & 56.76  \\
66 & -255.60 & -210.90 & -293.70 & -296.20 & -211.80 & -253.64 & 41.82  \\
67 & -185.70 & -329.10 & -244.70 & -269.90 & -259.90 & -257.86 & 51.49  \\
68 & -231.30 & -270.80 & -176.40 & -200.00 & -187.30 & -213.16 & 38.23  \\
69 & -269.00 & -259.00 & -196.60 & -162.80 & -212.70 & -220.02 & 44.14  \\
70 & -270.80 & -164.60 & -248.10 & -306.40 & -249.00 & -247.78 & 52.17  \\
71 & -224.50 & -140.10 & -255.60 & -222.70 & -211.00 & -210.78 & 42.82  \\
72 & -187.30 & -247.20 & -175.50 & -151.90 & -259.90 & -204.36 & 46.89  \\
73 & -233.80 & -211.80 & -257.40 & -222.70 & -210.00 & -227.14 & 19.42  \\
74 & -199.10 & -223.60 & -140.10 & -269.20 & -269.10 & -220.22 & 54.00  \\
75 & -256.30 & -294.60 & -283.50 & -244.70 & -223.60 & -260.54 & 28.81  \\
76 & -210.00 & -270.80 & -199.10 & -197.50 & -210.90 & -217.66 & 30.33  \\
77 & -140.10 & -269.00 & -302.80 & -235.40 & -244.70 & -238.40 & 60.82  \\
78 & -150.30 & -243.80 & -190.00 & -248.10 & -199.10 & -206.26 & 40.64  \\
79 & -209.30 & -233.80 & -258.10 & -187.30 & -224.50 & -222.60 & 26.52  \\
80 & -235.40 & -295.30 & -256.30 & -223.60 & -187.30 & -239.58 & 39.97  \\
81 & -184.80 & -200.00 & -211.80 & -174.60 & -210.10 & -196.26 & 16.18  \\
82 & -233.80 & -330.90 & -235.40 & -280.10 & -233.80 & -262.80 & 42.92  \\
83 & -209.10 & -234.50 & -200.90 & -174.60 & -199.20 & -203.66 & 21.52  \\
84 & -274.40 & -239.90 & -256.50 & -255.60 & -200.00 & -245.28 & 28.11  \\
85 & -280.80 & -199.10 & -285.30 & -235.40 & -232.80 & -246.68 & 36.19  \\
86 & -232.00 & -259.90 & -270.80 & -201.80 & -257.30 & -244.36 & 27.71  \\
87 & -173.00 & -259.00 & -210.90 & -281.70 & -200.00 & -224.92 & 44.44  \\
88 & -233.60 & -255.60 & -174.80 & -210.90 & -186.50 & -212.28 & 33.18  \\
89 & -223.60 & -259.00 & -282.60 & -221.80 & -281.70 & -253.74 & 29.88  \\
90 & -259.90 & -175.50 & -210.90 & -297.10 & -233.60 & -235.40 & 46.36  \\
91 & -249.00 & -244.70 & -256.50 & -200.00 & -198.50 & -229.74 & 28.16  \\
92 & -232.90 & -223.60 & -221.80 & -257.20 & -199.10 & -226.92 & 21.00  \\
93 & -200.90 & -196.60 & -141.00 & -210.90 & -186.60 & -187.20 & 27.26  \\
94 & -211.80 & -306.20 & -259.20 & -234.50 & -210.90 & -244.52 & 39.75  \\
95 & -316.40 & -235.40 & -186.40 & -232.20 & -233.60 & -240.80 & 46.98  \\
96 & -175.50 & -188.20 & -175.50 & -211.80 & -236.30 & -197.46 & 26.29  \\
97 & -294.40 & -222.00 & -221.80 & -186.40 & -255.60 & -236.04 & 40.78  \\
98 & -245.60 & -246.30 & -234.50 & -200.90 & -270.00 & -239.46 & 25.14  \\
99 & -200.00 & -234.50 & -210.00 & -175.50 & -224.50 & -208.90 & 22.88  \\
100 & -232.90 & -174.60 & -247.20 & -247.20 & -246.40 & -229.66 & 31.38  \\

\end{longtable}


\begin{longtable}{rrrrrrrr}
\resultcaption{realistic MBIE}{3}{3}
\resulthead

1 & -427.15 & -257.25 & -255.85 & -321.95 & -334.75 & -319.39 & 70.27  \\
2 & -271.55 & -408.95 & -497.75 & -407.45 & -283.05 & -373.75 & 95.42  \\
3 & -444.55 & -432.55 & -353.45 & -617.65 & -785.75 & -526.79 & 173.93  \\
4 & -613.85 & -499.55 & -566.45 & -456.75 & -397.45 & -506.81 & 85.93  \\
5 & -600.15 & -547.45 & -408.75 & -511.05 & -413.65 & -496.21 & 83.84  \\
6 & -708.35 & -388.05 & -576.75 & -439.35 & -488.95 & -520.29 & 126.07  \\
7 & -630.35 & -160.95 & -659.65 & -462.05 & -178.55 & -418.31 & 239.19  \\
8 & -502.65 & -178.75 & -597.75 & -656.25 & -251.35 & -437.35 & 211.77  \\
9 & -970.65 & -342.85 & -1\,100.55 & -547.25 & -195.25 & -631.31 & 392.35  \\
10 & -976.25 & -258.05 & -531.25 & -623.15 & -160.25 & -509.79 & 322.66  \\
11 & -612.85 & -248.65 & -346.45 & -185.65 & -174.05 & -313.53 & 180.76  \\
12 & -300.05 & -301.05 & -230.45 & -455.15 & -308.15 & -318.97 & 82.43  \\
13 & -310.45 & -209.65 & -245.35 & -112.45 & -185.65 & -212.71 & 73.16  \\
14 & -246.65 & -266.35 & -240.95 & -674.25 & -253.65 & -336.37 & 189.12  \\
15 & -317.05 & -184.95 & -301.05 & -231.65 & -250.55 & -257.05 & 53.44  \\
16 & -780.05 & -175.35 & -217.95 & -242.75 & -183.85 & -319.99 & 258.60  \\
17 & -209.05 & -234.55 & -288.25 & -252.75 & -312.65 & -259.45 & 41.43  \\
18 & -289.65 & -205.25 & -172.05 & -233.75 & -244.15 & -228.97 & 43.98  \\
19 & -217.35 & -207.15 & -319.45 & -194.25 & -221.25 & -231.89 & 50.05  \\
20 & -161.55 & -253.55 & -260.75 & -148.65 & -265.75 & -218.05 & 57.81  \\
21 & -216.85 & -245.85 & -253.15 & -233.45 & -242.15 & -238.29 & 13.93  \\
22 & -196.65 & -246.75 & -162.95 & -195.85 & -184.75 & -197.39 & 30.76  \\
23 & -317.05 & -199.45 & -388.95 & -452.25 & -242.95 & -320.13 & 103.35  \\
24 & -207.45 & -348.75 & -376.15 & -328.15 & -279.25 & -307.95 & 66.42  \\
25 & -277.95 & -302.75 & -344.75 & -206.05 & -183.15 & -262.93 & 67.28  \\
26 & -125.85 & -231.35 & -302.75 & -255.05 & -218.35 & -226.67 & 64.90  \\
27 & -230.95 & -297.65 & -257.65 & -333.45 & -217.65 & -267.47 & 47.90  \\
28 & -301.75 & -171.45 & -256.65 & -257.05 & -277.55 & -252.89 & 49.14  \\
29 & -257.35 & -467.65 & -374.25 & -550.15 & -331.75 & -396.23 & 114.80  \\
30 & -195.65 & -197.45 & -208.15 & -207.55 & -195.65 & -200.89 & 6.40  \\
31 & -207.45 & -255.05 & -241.55 & -231.95 & -241.45 & -235.49 & 17.70  \\
32 & -148.45 & -210.75 & -231.95 & -162.05 & -255.15 & -201.67 & 45.45  \\
33 & -195.65 & -389.95 & -265.15 & -185.65 & -217.55 & -250.79 & 83.61  \\
34 & -221.95 & -278.85 & -287.85 & -244.65 & -183.85 & -243.43 & 42.55  \\
35 & -162.25 & -308.15 & -245.55 & -255.65 & -237.35 & -241.79 & 52.34  \\
36 & -252.45 & -241.35 & -185.65 & -211.05 & -183.45 & -214.79 & 31.50  \\
37 & -244.25 & -207.65 & -209.25 & -406.95 & -136.75 & -240.97 & 100.67  \\
38 & -255.85 & -337.45 & -315.85 & -313.15 & -239.05 & -292.27 & 42.40  \\
39 & -162.05 & -209.55 & -258.25 & -280.05 & -232.15 & -228.41 & 45.64  \\
40 & -173.55 & -241.25 & -185.65 & -230.55 & -252.05 & -216.61 & 34.89  \\
41 & -242.75 & -284.95 & -269.15 & -210.15 & -524.35 & -306.27 & 125.17  \\
42 & -255.25 & -739.55 & -219.25 & -209.25 & -337.45 & -352.15 & 222.36  \\
43 & -255.55 & -171.75 & -282.95 & -270.05 & -268.85 & -249.83 & 44.71  \\
44 & -267.75 & -303.35 & -209.25 & -278.65 & -301.25 & -272.05 & 38.20  \\
45 & -244.65 & -230.65 & -212.85 & -279.85 & -586.95 & -310.99 & 156.21  \\
46 & -267.95 & -256.75 & -229.15 & -309.95 & -185.65 & -249.89 & 46.20  \\
47 & -267.55 & -256.15 & -221.95 & -279.15 & -217.85 & -248.53 & 27.41  \\
48 & -230.45 & -291.15 & -264.95 & -208.35 & -289.35 & -256.85 & 36.56  \\
49 & -227.55 & -195.05 & -221.95 & -437.65 & -196.55 & -255.75 & 102.73  \\
50 & -333.95 & -269.05 & -172.95 & -184.75 & -209.25 & -233.99 & 67.05  \\
51 & -242.55 & -303.85 & -242.45 & -247.35 & -232.85 & -253.81 & 28.46  \\
52 & -196.85 & -162.05 & -291.35 & -222.85 & -184.95 & -211.61 & 49.67  \\
53 & -171.75 & -314.65 & -190.45 & -234.65 & -221.15 & -226.53 & 55.14  \\
54 & -241.95 & -253.35 & -213.05 & -270.05 & -206.95 & -237.07 & 26.74  \\
55 & -278.65 & -184.05 & -158.05 & -351.75 & -303.45 & -255.19 & 81.70  \\
56 & -206.35 & -231.95 & -254.55 & -234.65 & -267.35 & -238.97 & 23.35  \\
57 & -197.45 & -147.95 & -236.75 & -148.45 & -279.45 & -202.01 & 57.04  \\
58 & -256.45 & -184.75 & -245.95 & -160.55 & -496.85 & -268.91 & 133.66  \\
59 & -209.05 & -302.95 & -257.85 & -186.55 & -229.25 & -237.13 & 45.20  \\
60 & -231.95 & -221.05 & -226.45 & -232.85 & -262.05 & -234.87 & 15.92  \\
61 & -197.75 & -292.05 & -225.85 & -186.55 & -181.05 & -216.65 & 45.55  \\
62 & -287.75 & -286.05 & -199.65 & -231.95 & -216.95 & -244.47 & 40.39  \\
63 & -195.65 & -278.75 & -255.35 & -221.95 & -181.65 & -226.67 & 40.45  \\
64 & -257.35 & -221.05 & -225.05 & -211.05 & -139.45 & -210.79 & 43.49  \\
65 & -269.35 & -218.55 & -226.55 & -234.65 & -264.35 & -242.69 & 22.85  \\
66 & -290.25 & -256.45 & -305.95 & -231.95 & -160.45 & -249.01 & 57.30  \\
67 & -162.95 & -196.55 & -289.15 & -232.85 & -249.25 & -226.15 & 48.52  \\
68 & -247.35 & -292.05 & -245.55 & -172.95 & -264.35 & -244.45 & 44.11  \\
69 & -210.15 & -533.45 & -261.25 & -237.35 & -214.65 & -291.37 & 136.85  \\
70 & -210.15 & -149.45 & -272.25 & -280.95 & -499.15 & -282.39 & 132.27  \\
71 & -162.95 & -290.35 & -218.15 & -207.45 & -160.25 & -207.83 & 52.90  \\
72 & -256.65 & -244.65 & -227.15 & -199.25 & -938.45 & -373.23 & 316.70  \\
73 & -184.75 & -173.85 & -203.95 & -245.55 & -267.15 & -215.05 & 39.95  \\
74 & -210.15 & -231.45 & -269.85 & -174.75 & -126.65 & -202.57 & 54.67  \\
75 & -255.55 & -327.25 & -276.35 & -233.75 & -233.45 & -265.27 & 38.94  \\
76 & -211.05 & -276.85 & -225.65 & -246.45 & -199.25 & -231.85 & 30.71  \\
77 & -198.35 & -221.15 & -192.85 & -161.15 & -282.15 & -211.13 & 45.11  \\
78 & -232.85 & -139.35 & -226.65 & -221.05 & -221.05 & -208.19 & 38.79  \\
79 & -242.85 & -219.45 & -224.95 & -199.25 & -221.05 & -221.51 & 15.55  \\
80 & -245.55 & -207.65 & -229.15 & -221.05 & -149.35 & -210.55 & 36.86  \\
81 & -135.95 & -209.35 & -350.15 & -245.55 & -244.65 & -237.13 & 77.31  \\
82 & -244.65 & -242.35 & -202.05 & -267.35 & -185.65 & -228.41 & 33.53  \\
83 & -256.45 & -199.25 & -226.55 & -211.95 & -174.75 & -213.79 & 30.50  \\
84 & -171.35 & -208.35 & -256.15 & -197.45 & -306.35 & -227.93 & 53.53  \\
85 & -160.25 & -187.45 & -243.65 & -275.45 & -256.45 & -224.65 & 48.69  \\
86 & -162.05 & -220.15 & -292.45 & -243.05 & -208.35 & -225.21 & 47.80  \\
87 & -326.35 & -208.45 & -205.55 & -232.55 & -264.45 & -247.47 & 50.03  \\
88 & -265.95 & -196.65 & -202.95 & -232.85 & -162.95 & -212.27 & 38.95  \\
89 & -257.35 & -139.35 & -123.35 & -221.05 & -208.55 & -189.93 & 56.68  \\
90 & -246.45 & -196.75 & -327.85 & -136.65 & -222.45 & -226.03 & 70.08  \\
91 & -208.35 & -301.85 & -224.15 & -256.65 & -233.05 & -244.81 & 36.36  \\
92 & -280.05 & -221.05 & -302.15 & -267.65 & -210.15 & -256.21 & 39.27  \\
93 & -150.25 & -184.75 & -254.05 & -221.05 & -219.55 & -205.93 & 39.62  \\
94 & -246.45 & -347.15 & -285.35 & -189.25 & -193.35 & -252.31 & 66.28  \\
95 & -255.55 & -257.45 & -233.75 & -207.45 & -127.55 & -216.35 & 53.60  \\
96 & -186.55 & -243.05 & -245.55 & -195.65 & -221.95 & -218.55 & 26.88  \\
97 & -245.55 & -174.75 & -245.55 & -232.85 & -269.15 & -233.57 & 35.40  \\
98 & -234.65 & -232.85 & -254.05 & -232.85 & -186.55 & -228.19 & 24.94  \\
99 & -222.85 & -172.95 & -234.65 & -235.55 & -127.55 & -198.71 & 47.33  \\
100 & -288.25 & -230.35 & -173.85 & -136.65 & -266.75 & -219.17 & 63.29  \\

\end{longtable}


\begin{longtable}{rrrrrrrr}
\resultcaption{realistic MBIE}{4}{3}
\resulthead

1 & -680.30 & -421.40 & -580.00 & -582.50 & -580.00 & -568.84 & 93.00  \\
2 & -421.50 & -478.80 & -575.30 & -631.50 & -575.30 & -536.48 & 84.54  \\
3 & -608.90 & -497.20 & -744.90 & -669.10 & -744.90 & -653.00 & 104.13  \\
4 & -1\,035.50 & -516.20 & -494.10 & -738.20 & -494.10 & -655.62 & 235.98  \\
5 & -556.40 & -992.10 & -781.50 & -368.60 & -781.50 & -696.02 & 239.26  \\
6 & -830.10 & -625.20 & -668.90 & -518.50 & -668.90 & -662.32 & 112.14  \\
7 & -574.90 & -645.70 & -660.60 & -417.40 & -660.60 & -591.84 & 103.77  \\
8 & -838.50 & -724.10 & -593.00 & -462.20 & -593.00 & -642.16 & 143.60  \\
9 & -832.00 & -647.20 & -485.10 & -487.80 & -485.10 & -587.44 & 153.51  \\
10 & -758.00 & -594.10 & -388.80 & -307.60 & -388.80 & -487.46 & 184.65  \\
11 & -476.40 & -676.40 & -463.00 & -660.90 & -463.00 & -547.94 & 110.46  \\
12 & -696.50 & -1\,287.10 & -843.80 & -473.40 & -843.80 & -828.92 & 297.62  \\
13 & -866.30 & -685.00 & -350.40 & -546.40 & -350.40 & -559.70 & 222.20  \\
14 & -292.00 & -332.50 & -452.40 & -608.20 & -452.40 & -427.50 & 123.77  \\
15 & -573.40 & -385.50 & -544.50 & -432.50 & -544.50 & -496.08 & 82.06  \\
16 & -416.10 & -995.90 & -1\,169.70 & -650.90 & -1\,169.70 & -880.46 & 335.03  \\
17 & -547.10 & -679.30 & -615.10 & -388.50 & -615.10 & -569.02 & 111.22  \\
18 & -713.10 & -545.40 & -536.70 & -576.10 & -536.70 & -581.60 & 75.27  \\
19 & -480.30 & -433.20 & -498.30 & -551.60 & -498.30 & -492.34 & 42.50  \\
20 & -847.90 & -571.00 & -540.60 & -812.40 & -540.60 & -662.50 & 154.06  \\
21 & -1\,118.30 & -390.80 & -518.70 & -560.70 & -518.70 & -621.44 & 284.98  \\
22 & -243.60 & -569.80 & -243.10 & -667.80 & -243.10 & -393.48 & 208.59  \\
23 & -433.70 & -620.90 & -359.40 & -334.80 & -359.40 & -421.64 & 117.41  \\
24 & -322.60 & -441.60 & -253.00 & -518.10 & -253.00 & -357.66 & 118.24  \\
25 & -308.60 & -323.90 & -365.90 & -194.20 & -365.90 & -311.70 & 70.43  \\
26 & -270.50 & -436.00 & -312.40 & -335.50 & -312.40 & -333.36 & 61.99  \\
27 & -276.00 & -437.00 & -371.40 & -277.50 & -371.40 & -346.66 & 69.21  \\
28 & -310.80 & -433.10 & -347.10 & -441.60 & -347.10 & -375.94 & 58.06  \\
29 & -251.40 & -530.80 & -294.60 & -290.20 & -294.60 & -332.32 & 112.43  \\
30 & -284.50 & -349.90 & -289.30 & -283.00 & -289.30 & -299.20 & 28.48  \\
31 & -305.60 & -230.30 & -312.30 & -334.60 & -312.30 & -299.02 & 39.95  \\
32 & -287.20 & -311.00 & -221.20 & -652.60 & -221.20 & -338.64 & 179.98  \\
33 & -309.20 & -338.40 & -207.60 & -290.50 & -207.60 & -270.66 & 60.04  \\
34 & -379.20 & -228.50 & -278.40 & -313.50 & -278.40 & -295.60 & 55.69  \\
35 & -354.30 & -302.80 & -338.30 & -232.30 & -338.30 & -313.20 & 49.00  \\
36 & -311.50 & -379.30 & -254.80 & -321.80 & -254.80 & -304.44 & 52.16  \\
37 & -276.80 & -322.90 & -345.80 & -384.60 & -345.80 & -335.18 & 39.46  \\
38 & -354.00 & -316.20 & -317.60 & -558.50 & -317.60 & -372.78 & 105.04  \\
39 & -265.50 & -240.10 & -312.80 & -331.00 & -312.80 & -292.44 & 38.02  \\
40 & -343.10 & -273.90 & -278.60 & -303.80 & -278.60 & -295.60 & 29.04  \\
41 & -320.20 & -253.70 & -265.70 & -372.80 & -265.70 & -295.62 & 50.27  \\
42 & -563.20 & -206.90 & -338.30 & -302.10 & -338.30 & -349.76 & 130.86  \\
43 & -325.80 & -305.60 & -426.30 & -376.40 & -426.30 & -372.08 & 55.81  \\
44 & -277.70 & -299.00 & -315.60 & -268.40 & -315.60 & -295.26 & 21.63  \\
45 & -244.90 & -265.00 & -219.40 & -244.80 & -219.40 & -238.70 & 19.44  \\
46 & -372.30 & -216.90 & -252.30 & -275.20 & -252.30 & -273.80 & 58.88  \\
47 & -221.30 & -334.90 & -292.00 & -347.10 & -292.00 & -297.46 & 49.31  \\
48 & -349.40 & -356.40 & -219.40 & -255.70 & -219.40 & -280.06 & 68.17  \\
49 & -298.00 & -243.10 & -369.00 & -315.60 & -369.00 & -318.94 & 52.95  \\
50 & -365.30 & -303.00 & -326.50 & -315.60 & -326.50 & -327.38 & 23.31  \\
51 & -347.10 & -287.70 & -254.80 & -305.60 & -254.80 & -290.00 & 38.69  \\
52 & -300.50 & -254.80 & -394.30 & -302.00 & -394.30 & -329.18 & 62.40  \\
53 & -298.90 & -316.50 & -207.60 & -327.10 & -207.60 & -271.54 & 59.23  \\
54 & -320.10 & -289.50 & -373.10 & -418.40 & -373.10 & -354.84 & 50.46  \\
55 & -287.90 & -334.90 & -241.20 & -266.60 & -241.20 & -274.36 & 39.08  \\
56 & -357.90 & -290.30 & -318.30 & -335.80 & -318.30 & -324.12 & 24.95  \\
57 & -301.60 & -288.60 & -294.70 & -242.10 & -294.70 & -284.34 & 24.06  \\
58 & -260.20 & -353.70 & -275.30 & -327.40 & -275.30 & -298.38 & 40.08  \\
59 & -239.80 & -264.10 & -312.30 & -184.90 & -312.30 & -262.68 & 53.62  \\
60 & -418.60 & -302.30 & -264.90 & -221.20 & -264.90 & -294.38 & 75.14  \\
61 & -278.60 & -324.70 & -267.80 & -266.60 & -267.80 & -281.10 & 24.86  \\
62 & -184.90 & -344.30 & -338.30 & -325.60 & -338.30 & -306.28 & 68.20  \\
63 & -343.90 & -335.00 & -279.30 & -267.50 & -279.30 & -301.00 & 35.57  \\
64 & -298.80 & -240.20 & -161.30 & -325.30 & -161.30 & -237.38 & 75.97  \\
65 & -220.40 & -330.20 & -344.40 & -230.30 & -344.40 & -293.94 & 62.98  \\
66 & -324.90 & -220.40 & -325.60 & -231.20 & -325.60 & -285.54 & 54.67  \\
67 & -345.80 & -372.10 & -316.50 & -207.80 & -316.50 & -311.74 & 62.55  \\
68 & -288.40 & -267.50 & -314.70 & -207.60 & -314.70 & -278.58 & 44.35  \\
69 & -298.10 & -253.00 & -300.90 & -359.60 & -300.90 & -302.50 & 37.87  \\
70 & -311.10 & -346.40 & -267.50 & -343.50 & -267.50 & -307.20 & 38.80  \\
71 & -288.70 & -325.70 & -243.90 & -339.20 & -243.90 & -288.28 & 44.53  \\
72 & -265.10 & -218.00 & -362.80 & -281.10 & -362.80 & -297.96 & 63.57  \\
73 & -337.10 & -383.00 & -281.10 & -329.20 & -281.10 & -322.30 & 42.85  \\
74 & -336.90 & -335.80 & -339.40 & -233.00 & -339.40 & -316.90 & 46.93  \\
75 & -277.30 & -291.20 & -313.10 & -292.00 & -313.10 & -297.34 & 15.53  \\
76 & -392.40 & -265.80 & -278.40 & -256.60 & -278.40 & -294.32 & 55.59  \\
77 & -276.20 & -276.90 & -266.60 & -300.40 & -266.60 & -277.34 & 13.82  \\
78 & -300.70 & -218.00 & -277.70 & -266.80 & -277.70 & -268.18 & 30.65  \\
79 & -347.90 & -240.50 & -280.20 & -269.30 & -280.20 & -283.62 & 39.43  \\
80 & -253.30 & -324.80 & -417.90 & -203.30 & -417.90 & -323.44 & 96.44  \\
81 & -355.40 & -339.20 & -281.10 & -359.50 & -281.10 & -323.26 & 39.23  \\
82 & -216.10 & -311.50 & -264.30 & -362.80 & -264.30 & -283.80 & 55.57  \\
83 & -220.40 & -325.70 & -315.60 & -382.10 & -315.60 & -311.88 & 58.13  \\
84 & -266.00 & -346.00 & -279.30 & -291.10 & -279.30 & -292.34 & 31.28  \\
85 & -1\,037.90 & -287.70 & -277.50 & -339.20 & -277.50 & -443.96 & 333.01  \\
86 & -288.60 & -288.40 & -372.80 & -339.20 & -372.80 & -332.36 & 42.32  \\
87 & -220.30 & -266.80 & -332.00 & -290.40 & -332.00 & -288.30 & 47.20  \\
88 & -206.10 & -242.30 & -348.70 & -245.70 & -348.70 & -278.30 & 66.11  \\
89 & -265.30 & -337.60 & -281.10 & -231.20 & -281.10 & -279.26 & 38.45  \\
90 & -277.90 & -279.70 & -344.60 & -231.20 & -344.60 & -295.60 & 48.77  \\
91 & -288.10 & -269.30 & -292.00 & -313.80 & -292.00 & -291.04 & 15.82  \\
92 & -277.00 & -300.50 & -279.30 & -337.40 & -279.30 & -294.70 & 25.71  \\
93 & -249.20 & -383.00 & -206.70 & -325.60 & -206.70 & -274.24 & 77.81  \\
94 & -209.50 & -265.00 & -266.60 & -325.60 & -266.60 & -266.66 & 41.06  \\
95 & -227.30 & -290.40 & -358.20 & -327.40 & -358.20 & -312.30 & 55.10  \\
96 & -356.10 & -243.10 & -302.00 & -253.90 & -302.00 & -291.42 & 45.14  \\
97 & -243.20 & -337.50 & -291.10 & -350.30 & -291.10 & -302.64 & 42.68  \\
98 & -281.70 & -277.70 & -300.40 & -266.60 & -300.40 & -285.36 & 14.80  \\
99 & -273.60 & -373.80 & -292.00 & -266.60 & -292.00 & -299.60 & 42.97  \\
100 & -240.60 & -289.80 & -323.40 & -287.90 & -323.40 & -293.02 & 34.02  \\

\end{longtable}


\begin{longtable}{rrrrrrrr}
\resultcaption{realistic MBIE}{5}{1}
\resulthead

1 & -830.10 & -731.70 & -1\,185.60 & -960.90 & -1\,067.90 & -955.24 & 181.26  \\
2 & -1\,065.80 & -891.60 & -669.40 & -429.90 & -398.80 & -691.10 & 289.28  \\
3 & -456.90 & -409.70 & -468.70 & -480.50 & -492.30 & -461.62 & 31.88  \\
4 & -551.30 & -433.30 & -421.50 & -456.90 & -468.70 & -466.34 & 51.03  \\
5 & -456.90 & -421.50 & -268.10 & -480.50 & -433.30 & -412.06 & 83.61  \\
6 & -362.50 & -492.30 & -409.70 & -504.10 & -362.50 & -426.22 & 68.60  \\
7 & -386.10 & -433.30 & -433.30 & -338.90 & -374.30 & -393.18 & 40.53  \\
8 & -338.90 & -362.50 & -338.90 & -433.30 & -327.10 & -360.14 & 42.87  \\
9 & -433.30 & -409.70 & -374.30 & -421.50 & -362.50 & -400.26 & 30.54  \\
10 & -433.30 & -327.10 & -338.90 & -456.90 & -409.70 & -393.18 & 57.57  \\
11 & -386.10 & -539.50 & -445.10 & -397.90 & -409.70 & -435.66 & 62.10  \\
12 & -551.30 & -445.10 & -397.90 & -350.70 & -338.90 & -416.78 & 86.15  \\
13 & -338.90 & -268.10 & -362.50 & -350.70 & -409.70 & -345.98 & 51.16  \\
14 & -338.90 & -468.70 & -610.30 & -421.50 & -362.50 & -440.38 & 107.70  \\
15 & -504.10 & -433.30 & -338.90 & -350.70 & -338.90 & -393.18 & 73.50  \\
16 & -492.30 & -456.90 & -563.10 & -338.90 & -445.10 & -459.26 & 81.50  \\
17 & -362.50 & -445.10 & -433.30 & -456.90 & -409.70 & -421.50 & 37.31  \\
18 & -504.10 & -433.30 & -480.50 & -445.10 & -468.70 & -466.34 & 28.17  \\
19 & -386.10 & -433.30 & -268.10 & -468.70 & -374.30 & -386.10 & 76.02  \\
20 & -409.70 & -480.50 & -362.50 & -350.70 & -303.50 & -381.38 & 67.06  \\
21 & -397.90 & -433.30 & -456.90 & -527.70 & -409.70 & -445.10 & 51.44  \\
22 & -504.10 & -480.50 & -445.10 & -433.30 & -397.90 & -452.18 & 41.38  \\
23 & -456.90 & -480.50 & -504.10 & -421.50 & -456.90 & -463.98 & 30.77  \\
24 & -468.70 & -456.90 & -421.50 & -386.10 & -362.50 & -419.14 & 45.24  \\
25 & -504.10 & -268.10 & -409.70 & -362.50 & -421.50 & -393.18 & 86.55  \\
26 & -409.70 & -492.30 & -433.30 & -397.90 & -350.70 & -416.78 & 51.84  \\
27 & -504.10 & -445.10 & -433.30 & -504.10 & -445.10 & -466.34 & 34.80  \\
28 & -433.30 & -421.50 & -397.90 & -409.70 & -374.30 & -407.34 & 22.70  \\
29 & -433.30 & -504.10 & -421.50 & -386.10 & -397.90 & -428.58 & 46.16  \\
30 & -386.10 & -386.10 & -350.70 & -386.10 & -492.30 & -400.26 & 53.69  \\
31 & -421.50 & -409.70 & -397.90 & -374.30 & -397.90 & -400.26 & 17.50  \\
32 & -492.30 & -480.50 & -397.90 & -492.30 & -350.70 & -442.74 & 64.85  \\
33 & -409.70 & -374.30 & -350.70 & -433.30 & -397.90 & -393.18 & 31.88  \\
34 & -515.90 & -527.70 & -386.10 & -421.50 & -362.50 & -442.74 & 75.28  \\
35 & -397.90 & -315.30 & -386.10 & -480.50 & -480.50 & -412.06 & 70.01  \\
36 & -504.10 & -515.90 & -456.90 & -480.50 & -386.10 & -468.70 & 51.44  \\
37 & -456.90 & -397.90 & -362.50 & -456.90 & -456.90 & -426.22 & 43.84  \\
38 & -350.70 & -350.70 & -563.10 & -327.10 & -480.50 & -414.42 & 102.73  \\
39 & -445.10 & -480.50 & -480.50 & -409.70 & -480.50 & -459.26 & 31.66  \\
40 & -386.10 & -456.90 & -468.70 & -421.50 & -386.10 & -423.86 & 38.60  \\
41 & -468.70 & -374.30 & -480.50 & -433.30 & -492.30 & -449.82 & 47.64  \\
42 & -563.10 & -397.90 & -397.90 & -397.90 & -539.50 & -459.26 & 84.43  \\
43 & -397.90 & -350.70 & -362.50 & -456.90 & -350.70 & -383.74 & 45.24  \\
44 & -386.10 & -374.30 & -409.70 & -291.70 & -468.70 & -386.10 & 64.09  \\
45 & -350.70 & -397.90 & -350.70 & -350.70 & -386.10 & -367.22 & 23.00  \\
46 & -421.50 & -504.10 & -386.10 & -362.50 & -574.90 & -449.82 & 88.15  \\
47 & -433.30 & -315.30 & -480.50 & -386.10 & -456.90 & -414.42 & 65.49  \\
48 & -468.70 & -374.30 & -433.30 & -315.30 & -409.70 & -400.26 & 58.64  \\
49 & -409.70 & -386.10 & -374.30 & -456.90 & -409.70 & -407.34 & 31.66  \\
50 & -445.10 & -386.10 & -456.90 & -445.10 & -362.50 & -419.14 & 42.05  \\
51 & -445.10 & -504.10 & -350.70 & -327.10 & -421.50 & -409.70 & 71.78  \\
52 & -409.70 & -338.90 & -362.50 & -338.90 & -492.30 & -388.46 & 64.85  \\
53 & -374.30 & -492.30 & -456.90 & -504.10 & -433.30 & -452.18 & 51.84  \\
54 & -433.30 & -374.30 & -421.50 & -433.30 & -433.30 & -419.14 & 25.58  \\
55 & -468.70 & -515.90 & -374.30 & -539.50 & -468.70 & -473.42 & 63.33  \\
56 & -586.70 & -338.90 & -409.70 & -421.50 & -504.10 & -452.18 & 95.35  \\
57 & -397.90 & -397.90 & -445.10 & -445.10 & -445.10 & -426.22 & 25.85  \\
58 & -374.30 & -433.30 & -386.10 & -504.10 & -504.10 & -440.38 & 62.22  \\
59 & -397.90 & -421.50 & -539.50 & -421.50 & -492.30 & -454.54 & 59.24  \\
60 & -350.70 & -303.50 & -421.50 & -397.90 & -315.30 & -357.78 & 51.16  \\
61 & -574.90 & -338.90 & -386.10 & -362.50 & -468.70 & -426.22 & 96.44  \\
62 & -374.30 & -445.10 & -268.10 & -374.30 & -433.30 & -379.02 & 70.11  \\
63 & -374.30 & -397.90 & -362.50 & -433.30 & -445.10 & -402.62 & 35.99  \\
64 & -386.10 & -480.50 & -386.10 & -456.90 & -386.10 & -419.14 & 46.00  \\
65 & -362.50 & -315.30 & -445.10 & -515.90 & -386.10 & -404.98 & 77.65  \\
66 & -468.70 & -338.90 & -315.30 & -456.90 & -421.50 & -400.26 & 69.51  \\
67 & -421.50 & -374.30 & -421.50 & -527.70 & -409.70 & -430.94 & 57.45  \\
68 & -421.50 & -374.30 & -433.30 & -504.10 & -504.10 & -447.46 & 56.22  \\
69 & -397.90 & -397.90 & -409.70 & -468.70 & -445.10 & -423.86 & 31.66  \\
70 & -468.70 & -456.90 & -362.50 & -563.10 & -468.70 & -463.98 & 71.09  \\
71 & -350.70 & -445.10 & -456.90 & -433.30 & -409.70 & -419.14 & 42.05  \\
72 & -492.30 & -504.10 & -386.10 & -504.10 & -421.50 & -461.62 & 54.46  \\
73 & -445.10 & -303.50 & -409.70 & -374.30 & -409.70 & -388.46 & 53.69  \\
74 & -338.90 & -327.10 & -456.90 & -445.10 & -374.30 & -388.46 & 59.82  \\
75 & -433.30 & -374.30 & -433.30 & -515.90 & -445.10 & -440.38 & 50.48  \\
76 & -445.10 & -409.70 & -374.30 & -504.10 & -456.90 & -438.02 & 49.08  \\
77 & -338.90 & -456.90 & -492.30 & -409.70 & -374.30 & -414.42 & 61.65  \\
78 & -515.90 & -327.10 & -504.10 & -303.50 & -468.70 & -423.86 & 100.96  \\
79 & -362.50 & -397.90 & -433.30 & -338.90 & -362.50 & -379.02 & 36.94  \\
80 & -409.70 & -374.30 & -409.70 & -421.50 & -445.10 & -412.06 & 25.58  \\
81 & -350.70 & -433.30 & -468.70 & -268.10 & -386.10 & -381.38 & 77.65  \\
82 & -480.50 & -433.30 & -433.30 & -327.10 & -468.70 & -428.58 & 60.51  \\
83 & -409.70 & -397.90 & -386.10 & -421.50 & -433.30 & -409.70 & 18.66  \\
84 & -397.90 & -433.30 & -327.10 & -480.50 & -468.70 & -421.50 & 61.88  \\
85 & -397.90 & -327.10 & -421.50 & -374.30 & -421.50 & -388.46 & 39.49  \\
86 & -409.70 & -327.10 & -445.10 & -397.90 & -338.90 & -383.74 & 49.64  \\
87 & -492.30 & -374.30 & -421.50 & -374.30 & -374.30 & -407.34 & 51.71  \\
88 & -374.30 & -515.90 & -480.50 & -397.90 & -492.30 & -452.18 & 62.22  \\
89 & -409.70 & -456.90 & -433.30 & -480.50 & -433.30 & -442.74 & 26.91  \\
90 & -445.10 & -397.90 & -374.30 & -421.50 & -480.50 & -423.86 & 41.22  \\
91 & -409.70 & -480.50 & -468.70 & -362.50 & -374.30 & -419.14 & 53.69  \\
92 & -468.70 & -374.30 & -374.30 & -374.30 & -362.50 & -390.82 & 43.84  \\
93 & -397.90 & -445.10 & -409.70 & -374.30 & -386.10 & -402.62 & 27.17  \\
94 & -362.50 & -504.10 & -350.70 & -409.70 & -350.70 & -395.54 & 65.38  \\
95 & -456.90 & -468.70 & -445.10 & -433.30 & -468.70 & -454.54 & 15.39  \\
96 & -374.30 & -421.50 & -338.90 & -374.30 & -386.10 & -379.02 & 29.62  \\
97 & -327.10 & -574.90 & -433.30 & -303.50 & -515.90 & -430.94 & 117.23  \\
98 & -468.70 & -480.50 & -445.10 & -456.90 & -456.90 & -461.62 & 13.45  \\
99 & -374.30 & -421.50 & -445.10 & -362.50 & -433.30 & -407.34 & 36.75  \\
100 & -433.30 & -445.10 & -386.10 & -362.50 & -421.50 & -409.70 & 34.40  \\

\end{longtable}


\begin{longtable}{rrrrrrrr}
\resultcaption{realistic MBIE}{5}{3}
\resulthead

1 & -739.95 & -401.95 & -462.55 & -452.65 & -487.75 & -508.97 & 132.84  \\
2 & -579.65 & -804.45 & -755.95 & -977.85 & -301.95 & -683.97 & 256.36  \\
3 & -906.05 & -819.35 & -816.65 & -588.75 & -544.65 & -735.09 & 158.64  \\
4 & -821.05 & -1\,032.25 & -777.05 & -924.15 & -878.05 & -886.51 & 98.74  \\
5 & -837.95 & -833.15 & -835.75 & -921.05 & -1\,090.55 & -903.69 & 110.83  \\
6 & -805.55 & -1\,118.35 & -880.15 & -556.35 & -924.25 & -856.93 & 203.96  \\
7 & -702.45 & -998.35 & -309.95 & -1\,315.65 & -584.25 & -782.13 & 387.25  \\
8 & -810.35 & -670.55 & -607.05 & -1\,642.05 & -611.25 & -868.25 & 440.31  \\
9 & -960.95 & -871.05 & -793.85 & -702.35 & -653.85 & -796.41 & 124.32  \\
10 & -1\,130.65 & -521.65 & -1\,431.75 & -621.05 & -815.75 & -904.17 & 375.48  \\
11 & -550.75 & -638.05 & -1\,053.45 & -761.55 & -691.35 & -739.03 & 191.89  \\
12 & -844.85 & -700.75 & -777.75 & -500.15 & -652.65 & -695.23 & 131.43  \\
13 & -487.45 & -884.55 & -664.15 & -584.55 & -649.85 & -654.11 & 146.49  \\
14 & -478.85 & -1\,028.85 & -2\,106.15 & -619.15 & -1\,717.95 & -1\,190.19 & 702.79  \\
15 & -355.05 & -712.95 & -767.15 & -881.55 & -780.35 & -699.41 & 201.92  \\
16 & -367.15 & -969.95 & -536.35 & -616.45 & -963.95 & -690.77 & 267.71  \\
17 & -863.85 & -749.65 & -700.05 & -638.35 & -735.45 & -737.47 & 82.69  \\
18 & -453.85 & -635.45 & -506.05 & -515.25 & -589.55 & -540.03 & 72.03  \\
19 & -437.55 & -955.45 & -669.45 & -593.65 & -828.85 & -696.99 & 201.83  \\
20 & -403.15 & -492.15 & -547.35 & -646.25 & -573.95 & -532.57 & 91.16  \\
21 & -286.95 & -599.85 & -648.75 & -685.85 & -288.95 & -502.07 & 197.83  \\
22 & -301.25 & -594.85 & -630.45 & -490.85 & -554.55 & -514.39 & 129.94  \\
23 & -297.85 & -728.85 & -783.35 & -1\,354.75 & -333.95 & -699.75 & 427.84  \\
24 & -624.95 & -894.45 & -678.95 & -482.15 & -416.05 & -619.31 & 186.69  \\
25 & -312.75 & -973.55 & -518.95 & -447.75 & -403.75 & -531.35 & 258.24  \\
26 & -299.75 & -1\,320.25 & -784.25 & -898.25 & -229.55 & -706.41 & 450.67  \\
27 & -371.25 & -750.95 & -735.65 & -315.35 & -396.65 & -513.97 & 211.47  \\
28 & -391.55 & -335.95 & -683.25 & -335.05 & -397.45 & -428.65 & 145.37  \\
29 & -333.25 & -367.45 & -965.75 & -286.95 & -321.65 & -455.01 & 286.96  \\
30 & -383.15 & -323.95 & -896.75 & -392.45 & -339.35 & -467.13 & 241.88  \\
31 & -300.55 & -301.25 & -1\,700.65 & -458.65 & -431.95 & -638.61 & 598.15  \\
32 & -361.25 & -475.75 & -1\,039.65 & -244.05 & -254.05 & -474.95 & 329.37  \\
33 & -335.95 & -384.75 & -659.55 & -347.75 & -367.45 & -419.09 & 135.71  \\
34 & -302.25 & -312.35 & -979.25 & -418.35 & -344.35 & -471.31 & 287.57  \\
35 & -336.75 & -336.65 & -1\,060.25 & -417.65 & -432.95 & -516.85 & 307.03  \\
36 & -456.55 & -356.65 & -431.15 & -369.25 & -267.65 & -376.25 & 73.64  \\
37 & -347.65 & -650.85 & -323.05 & -408.35 & -349.35 & -415.85 & 135.06  \\
38 & -357.85 & -348.45 & -344.45 & -318.25 & -362.95 & -346.39 & 17.36  \\
39 & -361.75 & -349.35 & -276.05 & -404.95 & -365.65 & -351.55 & 47.06  \\
40 & -344.45 & -409.05 & -490.25 & -346.15 & -359.65 & -389.91 & 61.91  \\
41 & -346.95 & -332.05 & -337.55 & -297.95 & -380.95 & -339.09 & 29.83  \\
42 & -325.75 & -382.95 & -288.45 & -310.55 & -359.45 & -333.43 & 37.86  \\
43 & -337.85 & -405.85 & -387.45 & -301.35 & -335.05 & -353.51 & 42.42  \\
44 & -263.15 & -380.45 & -311.25 & -300.45 & -373.85 & -325.83 & 50.19  \\
45 & -404.25 & -360.25 & -273.05 & -326.75 & -473.45 & -367.55 & 76.17  \\
46 & -298.65 & -464.85 & -323.95 & -403.15 & -446.05 & -387.33 & 73.47  \\
47 & -417.75 & -370.25 & -419.05 & -422.35 & -336.65 & -393.21 & 38.23  \\
48 & -277.75 & -360.45 & -416.95 & -311.35 & -338.45 & -340.99 & 52.52  \\
49 & -342.75 & -310.55 & -346.85 & -452.05 & -430.15 & -376.47 & 61.14  \\
50 & -441.25 & -419.45 & -736.15 & -436.95 & -334.85 & -473.73 & 152.89  \\
51 & -393.95 & -381.55 & -361.15 & -323.95 & -360.25 & -364.17 & 26.60  \\
52 & -333.45 & -372.05 & -441.25 & -406.45 & -338.85 & -378.41 & 45.72  \\
53 & -393.25 & -350.45 & -323.25 & -310.55 & -251.55 & -325.81 & 52.22  \\
54 & -431.15 & -313.95 & -360.25 & -404.95 & -396.55 & -381.37 & 45.43  \\
55 & -359.65 & -394.05 & -266.55 & -359.65 & -382.05 & -352.39 & 50.22  \\
56 & -291.35 & -290.35 & -349.45 & -382.05 & -420.45 & -346.73 & 56.87  \\
57 & -355.45 & -391.55 & -276.85 & -365.75 & -358.05 & -349.53 & 43.07  \\
58 & -361.35 & -382.25 & -335.25 & -476.85 & -337.55 & -378.65 & 58.16  \\
59 & -462.45 & -454.05 & -332.45 & -358.95 & -395.85 & -400.75 & 57.19  \\
60 & -310.55 & -371.15 & -337.65 & -370.85 & -347.65 & -347.57 & 25.33  \\
61 & -357.85 & -427.45 & -413.45 & -276.75 & -374.75 & -370.05 & 59.28  \\
62 & -405.95 & -394.35 & -313.15 & -343.15 & -335.25 & -358.37 & 39.90  \\
63 & -419.45 & -267.65 & -357.75 & -297.85 & -299.85 & -328.51 & 60.40  \\
64 & -297.05 & -407.45 & -358.75 & -264.35 & -358.45 & -337.21 & 56.50  \\
65 & -346.85 & -345.55 & -372.05 & -358.45 & -410.15 & -366.61 & 26.58  \\
66 & -370.35 & -394.25 & -345.45 & -393.15 & -428.85 & -386.41 & 31.00  \\
67 & -467.55 & -427.65 & -394.75 & -322.65 & -431.95 & -408.91 & 54.69  \\
68 & -360.75 & -441.35 & -338.45 & -444.65 & -389.45 & -394.93 & 47.47  \\
69 & -393.25 & -419.25 & -324.35 & -324.95 & -290.45 & -350.45 & 53.61  \\
70 & -357.55 & -369.85 & -478.35 & -418.25 & -322.15 & -389.23 & 60.55  \\
71 & -432.05 & -310.55 & -300.35 & -369.75 & -348.45 & -352.23 & 52.73  \\
72 & -437.05 & -334.75 & -345.75 & -323.35 & -457.85 & -379.75 & 62.74  \\
73 & -372.05 & -381.65 & -385.65 & -437.85 & -442.25 & -403.89 & 33.41  \\
74 & -359.85 & -333.55 & -392.25 & -347.75 & -367.35 & -360.15 & 22.04  \\
75 & -405.85 & -368.55 & -323.95 & -334.05 & -375.05 & -361.49 & 33.03  \\
76 & -360.45 & -1\,571.25 & -301.55 & -335.05 & -361.15 & -585.89 & 551.37  \\
77 & -336.65 & -384.95 & -251.35 & -323.35 & -280.35 & -315.33 & 51.68  \\
78 & -425.45 & -331.65 & -411.45 & -463.95 & -431.95 & -412.89 & 49.32  \\
79 & -357.75 & -322.35 & -524.75 & -423.55 & -325.75 & -390.83 & 85.18  \\
80 & -287.95 & -370.75 & -313.05 & -384.95 & -345.95 & -340.53 & 40.11  \\
81 & -360.45 & -335.75 & -313.95 & -314.85 & -303.05 & -325.61 & 22.79  \\
82 & -394.15 & -456.45 & -418.75 & -345.95 & -336.95 & -390.45 & 50.03  \\
83 & -383.15 & -254.95 & -373.85 & -464.95 & -347.75 & -364.93 & 75.51  \\
84 & -322.25 & -308.95 & -325.75 & -385.65 & -636.75 & -395.87 & 137.86  \\
85 & -383.95 & -453.35 & -394.85 & -349.65 & -347.75 & -385.91 & 43.02  \\
86 & -357.95 & -408.35 & -325.75 & -336.85 & -313.95 & -348.57 & 37.14  \\
87 & -394.75 & -396.55 & -381.35 & -440.55 & -332.55 & -389.15 & 38.71  \\
88 & -392.85 & -301.25 & -391.75 & -317.55 & -455.65 & -371.81 & 62.83  \\
89 & -373.85 & -393.85 & -323.25 & -379.75 & -394.25 & -372.99 & 29.19  \\
90 & -372.05 & -431.25 & -327.55 & -251.85 & -430.15 & -362.57 & 75.60  \\
91 & -393.85 & -385.65 & -383.05 & -444.05 & -310.35 & -383.39 & 47.77  \\
92 & -942.95 & -345.05 & -397.95 & -359.55 & -347.75 & -478.65 & 260.41  \\
93 & -333.95 & -312.15 & -345.05 & -410.15 & -348.45 & -349.95 & 36.52  \\
94 & -359.55 & -309.75 & -372.05 & -384.15 & -335.45 & -352.19 & 29.80  \\
95 & -384.75 & -314.15 & -359.55 & -312.15 & -253.15 & -324.75 & 50.52  \\
96 & -372.95 & -311.55 & -336.95 & -439.45 & -311.45 & -354.47 & 53.77  \\
97 & -321.15 & -394.05 & -406.05 & -339.35 & -336.35 & -359.39 & 37.99  \\
98 & -432.25 & -379.25 & -346.15 & -335.05 & -410.15 & -380.57 & 41.24  \\
99 & -335.45 & -324.15 & -393.35 & -430.45 & -398.35 & -376.35 & 44.99  \\
100 & -386.55 & -374.75 & -369.65 & -325.75 & -387.45 & -368.83 & 25.26  \\

\end{longtable}


\subsection{Factored additions to \etre}
\label{sec:factored_e3}

The \etre\ algorithm does not exploit that the underlying Markov Decision Process may be structured in a way that allows certain optimizations. Therefore \etre\ has a running time that scales polynomially with the number of states in the MDP. However, by using a factored approach for the problem, improvements can be made to the running time. By factoring the problem as a dynamic Bayesian network, the running time will scale with the number of random variables in the underlying DBN instead for the number of states \parencite{kearns1999efficient}. 

When using a factored representation some changes to the original algorithm are required to make it compatible. One issue that has to be solved is how to perform  planning with the new representation. This thesis a modified version of value iteration was used for planning and it is described later in this section. In section \ref{sec:better_planing_algos} there are other methods presented.

\paragraph{Dynamic Bayesian network structure}

Assume that the states of an MDP each are divided into several variables. For
instance, the invasive species MDP described in section
\ref{sec:experiment_env} constitutes such a case, where the status of each
reach can be considered a variable on its own. The number of tamarisk trees,
native trees and empty slots in a certain reach at time step $t+1$ depends not
on the whole state of the environment at time $t$, but only on the status of
adjacent reaches. Those variables on which another variable depend are called
its parents.  

An MDP that follows the description in the previous paragraph is described as
factored. With the assumption of a factored MDP, it is possible to describe its
transition probabilities as a dynamic Bayesian network, where one would have a
small transition probability table for each of the reaches in the MDP, instead
of a large table for the transitions for the whole states.

\paragraph{Planning in dynamic Bayesian networks}

The DBN-\etre\ algorithm does not in itself define what algorithm should be
used for planning when the MDP is structured as a DBN
\parencite{kearns1999efficient}. It considers planning a black box, leaving the
choice of planning algorithm to the implementers. 

Value iteration can be done with a factored representation of an MDP in a fairly straightforward manner. The same equations that normal value iteration (section~\ref{sec:valueiteration}) is based on can be used when the MDP is factored too. The only difference is that in order to calculate the probability of a state transition, $p(s'| a, s)$ one has to find the product of all the partial transitions,
\begin{equation}
\prod\limits _{i} p(s_i' | a, s_{pa(i)})
\end{equation}
where $i$ ranges over all partial states and $s_{pa(i)}$ is the setting of the partial states that have an influence on the value of $s_i'$. 

When an MDP has this structure, observations of partial transitions can be pooled together when the state variables are part of similar structures in the MDP. In the version of DBN-\etre\ described here, all state variables that have the same number of parent variables have their observations pooled together. 

\paragraph{One policy per state variable}
\label{sec:one_policy_per_state_variable}

For some MDPs it is possible to compute a separate policy for each state
variable individually. This is the case when there is a separate action taken
for each state variable, which is true for the Invasive Species environment
(section \ref{sec:experiment_env}). In the implementation of \etre\ used in
this thesis, this policy computation is performed in two steps. 

In the first step, a policy is calculated for state variables that have no
other state variables than themselves as parents in the DBN, and these states
are marked as done. This calculation is done by value iteration where the reward function
is described in \refeq{equation:fusk} and described below. Since there is now a decided action for each value for
these state variables, the transition probabilities for these variables can be
considered as pure Markov chains in the next step. In this step, a policy is
found for state variables whose parents are marked as done, until all state
variables are done.  In this second step, the transition probabilities of the
parents are thus treated as independent of the action taken.

The reward function for the partial action $a(i)$ and the partial state $s(i)$ in the Invasive Species environment can be described as follows:

\begin{equation}
\label{equation:fusk}
R(s(i),a(i)) = c(s(i)) r_c + t(s(i)) r_t + n(s(i)) r_n + e(s(i)) r_e +  x(a(i)) t(s(i)) r_x +  p(a(i)) e(s(i)) r_p
\end{equation}

Here $c(s(i))$ is equal to 1 if $s(i)$ is infected and 0 otherwise, $r_c$ is the reward given for each infected reach, $t(s(i))$ is the number of tamarix-infested habitats in $s(i)$, $r_t$ is the reward given for each tamarix habitat, $n(s(i))$ is the number of habitats with native trees in $s(i)$, $r_n$ is the reward given for each native habitat, $e(s(i))$ is the number of empty habitats in $s(i)$, $e_r$ is the reward given for empty habitats, $x(a(i))$ is equal to 1 if the action taken is to exterminate tamarix trees and otherwise 0, $r_x$ is the reward for trying to remove a tamarix tree, $p(a(i))$ is equal to one if the action taken is to plant native trees and 0 otherwise, and $r_p$ is the reward for trying to plant a native tree. 
Since \eqref{equation:fusk} is a simple linear equation, the unknown variables ($r_i, r_t, r_x$ and $r_p$) can be calculated exactly once a few data points have been collected. Once this is done, the agent can use \eqref{equation:fusk} to calculate the reward for any partial state-action pair. 

Planning for each state variable individually has the benefit of making the
planning algorithm linear in the number of state variables, greatly reducing
the time needed to calculate a policy. However, there are several downsides to
using this kind of approximation, some of which are discussed in section
\ref{sec:e3_factored_discussion}. 

