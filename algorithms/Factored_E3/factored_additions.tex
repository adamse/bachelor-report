\subsection{Our contribution}
\paragraph{Factored additions to \etre}
\label{sec:factored_e3}

The \etre\ algorithm does not exploit that the underlying Markov Decision Process may be structured in a way that allows certain optimizations. Therefore \etre\ has a running time that scales polynomially with the number of states in the MDP. However, by using a factored approach for the problem, improvements can be made to the running time. By factoring the problem as a dynamic Bayesian network, the running time will scale with the number of random variables in the underlying DBN instead for the number of states \parencite{kearns1999efficient}. 

When using a factored representation some changes to the original algorithm are required to make it compatible. One issue that has to be solved is how to perform  planning with the new representation. This thesis a modified version of value iteration was used for planning and it is described later in this section. In section \ref{sec:better_planing_algos} there are other methods presented.

\paragraph{Dynamic Bayesian network structure}

Assume that the states of an MDP each are divided into several variables. For
instance, the invasive species MDP described in section
\ref{sec:experiment_env} constitutes such a case, where the status of each
reach can be considered a variable on its own. The number of tamarisk trees,
native trees and empty slots in a certain reach at time step $t+1$ depends not
on the whole state of the environment at time $t$, but only on the status of
adjacent reaches. Those variables on which another variable depend are called
its parents.  

An MDP that follows the description in the previous paragraph is described as
factored. With the assumption of a factored MDP, it is possible to describe its
transition probabilities as a dynamic Bayesian network, where one would have a
small transition probability table for each of the reaches in the MDP, instead
of a large table for the transitions for the whole states.

\paragraph{Planning in dynamic Bayesian networks}

The DBN-\etre\ algorithm does not in itself define what algorithm should be
used for planning when the MDP is structured as a DBN
\parencite{kearns1999efficient}. It considers planning a black box, leaving the
choice of planning algorithm to the implementers. 

Value iteration can be done with a factored representation of an MDP in a fairly straightforward manner. The same equations that normal value iteration (section~\ref{sec:valueiteration}) is based on can be used when the MDP is factored too. The only difference is that in order to calculate the probability of a state transition, $p(s'| a, s)$ one has to find the product of all the partial transitions,
\begin{equation}
\prod\limits _{i} p(s_i' | a, s_{pa(i)})
\end{equation}
where $i$ ranges over all partial states and $s_{pa(i)}$ is the setting of the partial states that have an influence on the value of $s_i'$. 

When an MDP has this structure, observations of partial transitions can be pooled together when the state variables are part of similar structures in the MDP. In the version of DBN-\etre\ described here, all state variables that have the same number of parent variables have their observations pooled together. 
