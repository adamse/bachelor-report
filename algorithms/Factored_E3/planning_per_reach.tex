\subsection{One policy per state variable}
\label{sec:one_policy_per_state_variable}

For some MDPs it is possible to create a separate policy for each state
variable individually. This is the case when there is a separate action taken
for each state variable, which is true for the invasive species environment
(section \ref{sec:experiment_env}). In the implementation of \etre\ used in
this thesis, this policy creation is performed in two steps. 

First, a policy is calculated for state variables that have no other state
variables than themselves as parents in the DBN, and these states are marked as
done. Since there is now a decided action for each state for the these state
variables, the transition probabilities for these variables can be considered
as pure Markov chains. Next, a policy is found for state variables whose
parents are marked as done, until all state variables are done. In this second
step, the transition probabilities of the parents are thus treated as
independent of the action taken. 

Planning for each state variable individually has the benefit of making the
planning algorithm linear in the number of state variables, greatly reducing
the time until an agent can enter the exploitation phase for large state
spaces. However, there are several downsides to using this kind of
approximation, some of which are discussed in section. \ref{..}. \todo{Fix the
reference!}
