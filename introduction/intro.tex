Reinforcement learning, a subfield of artificial intelligence, is the study of algorithms that learn how to choose the best actions depending on the situation. In a reinforcement learning problem the algorithm is not told which actions give the best results, but instead it has to interact with the environment to learn when and where to take a certain action. When the agent has learned about each situation or state of the environment, it will arrive at an optimal sequence of actions \parencite{barto1998reinforcement}.

%Formally described in reinforcement learning as choosing an action that, depending on the state of the environment, yields the greatest reward.

%Reinforcement learning, a subfield of artificial intelligence, is the study of algorithms that learn how to choose the best actions depending on the situation, that is, the state of the environment. The best action or sequence of actions is what leads to the best results, that is to say, the greatest rewards. In a reinforcement learning problem the algorithm is not told which actions maximize the reward, but instead it has to interact with the environment to learn when and where to take a certain action. When the agent has learned about each situation or state of the environment, it will arrive at an optimal sequence of actions \parencite{barto1998reinforcement}.

% The field of reinforcement learning, a subfield of Artificial Intelligence, is the field of learning how to choose the best action in a certain situation. The best action or sequence of actions is what leads to the best results, that is to say, the greatest rewards. In a reinforcement learning problem we are not told which actions maximize the reward, instead the agent has to interact with the current environment to learn when and where to take a certain action. Learning about each situation or state yields an optimal sequence of actions. \parencite{barto1998reinforcement}.

Reinforcement learning can be exemplified by how a newborn animal learns how to stand up. There is no tutor to teach it, so instead it tries various combinations of movements, while remembering which of them lead to success and which of them lead to failure. Probably, at first, most of its attempted movement patterns will lead to it falling down --- a negative result, making the animal less likely to try those patterns again. After a while, the animal finds some combination of muscle contractions that enables it to stand --- a positive result, which would make it more likely to perform those actions again. 

In the real world, in order to decide what actions lead to success, one almost always needs to consider the circumstances, or states, in which the actions are taken. For instance, a person might be rewarded if they sang beautifully at a concert, but doing the same at the library would probably get them thrown out. Furthermore, for many real world domains it is common that the state space (the set of possible states of the environments in which actions can be taken) for the reinforcement learning problem becomes very large \parencite{guestrin2003efficient}. To continue our example, there are probably countless places and situations where it is possible to sing. In this case, how does one abstract and find the qualities of the environment that are important for deciding on the optimal action?

%The circumstance, or the state, of your environment determines the result that your actions have. 

There are two practical problems connected to reinforcement learning in large state spaces. First, it is hard to store representations of the states of the environment in the main memory of a computer.  \parencite{szepesvari2010algorithms}. Second, it takes too long to repeatedly visit all the states to find the best action to take in all of them \parencite{dietterich2013pac}.
